%% Generated by Sphinx.
\def\sphinxdocclass{report}
\documentclass[letterpaper,10pt,english]{sphinxmanual}
\ifdefined\pdfpxdimen
   \let\sphinxpxdimen\pdfpxdimen\else\newdimen\sphinxpxdimen
\fi \sphinxpxdimen=.75bp\relax

\PassOptionsToPackage{warn}{textcomp}
\usepackage[utf8]{inputenc}
\ifdefined\DeclareUnicodeCharacter
% support both utf8 and utf8x syntaxes
  \ifdefined\DeclareUnicodeCharacterAsOptional
    \def\sphinxDUC#1{\DeclareUnicodeCharacter{"#1}}
  \else
    \let\sphinxDUC\DeclareUnicodeCharacter
  \fi
  \sphinxDUC{00A0}{\nobreakspace}
  \sphinxDUC{2500}{\sphinxunichar{2500}}
  \sphinxDUC{2502}{\sphinxunichar{2502}}
  \sphinxDUC{2514}{\sphinxunichar{2514}}
  \sphinxDUC{251C}{\sphinxunichar{251C}}
  \sphinxDUC{2572}{\textbackslash}
\fi
\usepackage{cmap}
\usepackage[T1]{fontenc}
\usepackage{amsmath,amssymb,amstext}
\usepackage{babel}



\usepackage{times}
\expandafter\ifx\csname T@LGR\endcsname\relax
\else
% LGR was declared as font encoding
  \substitutefont{LGR}{\rmdefault}{cmr}
  \substitutefont{LGR}{\sfdefault}{cmss}
  \substitutefont{LGR}{\ttdefault}{cmtt}
\fi
\expandafter\ifx\csname T@X2\endcsname\relax
  \expandafter\ifx\csname T@T2A\endcsname\relax
  \else
  % T2A was declared as font encoding
    \substitutefont{T2A}{\rmdefault}{cmr}
    \substitutefont{T2A}{\sfdefault}{cmss}
    \substitutefont{T2A}{\ttdefault}{cmtt}
  \fi
\else
% X2 was declared as font encoding
  \substitutefont{X2}{\rmdefault}{cmr}
  \substitutefont{X2}{\sfdefault}{cmss}
  \substitutefont{X2}{\ttdefault}{cmtt}
\fi


\usepackage[Bjarne]{fncychap}
\usepackage[,numfigreset=1,mathnumfig]{sphinx}

\fvset{fontsize=\small}
\usepackage{geometry}


% Include hyperref last.
\usepackage{hyperref}
% Fix anchor placement for figures with captions.
\usepackage{hypcap}% it must be loaded after hyperref.
% Set up styles of URL: it should be placed after hyperref.
\urlstyle{same}


\usepackage{sphinxmessages}




\title{Engineering Dynamics}
\date{Nov 16, 2020}
\release{}
\author{Ryan C.\@{} Cooper}
\newcommand{\sphinxlogo}{\vbox{}}
\renewcommand{\releasename}{}
\makeindex
\begin{document}

\pagestyle{empty}
\sphinxmaketitle
\pagestyle{plain}
\sphinxtableofcontents
\pagestyle{normal}
\phantomsection\label{\detokenize{intro::doc}}


This
work is licensed under a Creative Commons
Attribution 4.0 International License.



Applied Mechanics II (or Dynamics) is the study of how things move and interact.
We are limiting our study to Newtonian mechanics. We won’t consider quantum
effects like wave\sphinxhyphen{}particle duality or relativistic effects. Our current interest
is to describe free and constrained motion much less than the speed of light and
with mass much larger than an atom, but much smaller than the sun (\textasciitilde{}1e\sphinxhyphen{}27 \textless{} m \textless{}
2e30 kg).

The two founding scientists of classical dynamics are \sphinxhref{https://en.wikipedia.org/wiki/Galileo\_Galilei}{Galileo
Galilei} (1564\sphinxhyphen{}1642) and \sphinxhref{https://en.wikipedia.org/wiki/Isaac\_Newton}{Isaac
Newton} (1642\sphinxhyphen{}1726). Galileo
described the geometry of moving objects and helped define our understanding of
kinematics. Newton defined three kinetic laws that helped describe how force,
impact, and energy relate to changes in motion.

This course explores the geometry of motion and the effect of forces,
impact, and energy in engineering. \sphinxstylestrong{Kinematics} is the study of the
geometry of motion. \sphinxstylestrong{Kinetics} is the study of forces, impacts, and
energy on objects.




\chapter{About the Instructor}
\label{\detokenize{module_00/meet_cooper:about-the-instructor}}\label{\detokenize{module_00/meet_cooper::doc}}
Ryan C. Cooper, Ph.D.Assistant Professor\sphinxhyphen{}in\sphinxhyphen{}ResidenceMechanical Engineering DepartmentUniversity of Connecticut191 Auditorium rdEngineering Building II room 314Storrs, CT 06269

\sphinxstylestrong{email:} \sphinxurl{mailto:ryan.c.cooper@uconn.edu}

\sphinxstylestrong{Office hours:} Mon, Wed, Thu 9\sphinxhyphen{}11am

\sphinxstylestrong{Preferred contact:} post discussion questions on Piazza, for personal
questions reach out via email.


\chapter{Module Content}
\label{\detokenize{content:module-content}}\label{\detokenize{content::doc}}
This textbook has one introduction module and five learning modules.
Each learning module includes:
\begin{enumerate}
\sphinxsetlistlabels{\arabic}{enumi}{enumii}{}{.}%
\item {} 
one reading assignment with discussion in Piazza

\item {} 
lecture and tutorial videos

\item {} 
one homework assigment

\item {} 
one quiz based upon the homework, readings, and videos

\item {} 
one Adams report

\end{enumerate}

Make sure you reflect on things you learned, what you found easy, and
what you struggled with after each module in the \sphinxstylestrong{Summary and
Checklist}.


\chapter{Module 0 \sphinxhyphen{} Getting Started}
\label{\detokenize{module_00/overview:module-0-getting-started}}\label{\detokenize{module_00/overview::doc}}
This introduction will get you up to speed on the tools we are using to
explore dynamics: HuskyCT, Google Documents, Adams multibody dynamics,
and Piazza discussion boar4.


\section{Overview}
\label{\detokenize{module_00/introduction:overview}}\label{\detokenize{module_00/introduction::doc}}


Applied Mechanics II (or Dynamics) is the study of how things move and interact.
We are limiting our study to Newtonian mechanics. We won’t consider quantum
effects like wave\sphinxhyphen{}particle duality or relativistic effects. Our current interest
is to describe free and constrained motion much less than the speed of light and
with mass much larger than an atom, but much smaller than the sun (\textasciitilde{}1e\sphinxhyphen{}27 \textless{} m \textless{}
2e30 kg).

The two founding scientists of classical dynamics are \sphinxhref{https://en.wikipedia.org/wiki/Galileo\_Galilei}{Galileo
Galilei} (1564\sphinxhyphen{}1642) and \sphinxhref{https://en.wikipedia.org/wiki/Isaac\_Newton}{Isaac
Newton} (1642\sphinxhyphen{}1726). Galileo
described the geometry of moving objects and helped define our understanding of
kinematics. Newton defined three kinetic laws that helped describe how force,
impact, and energy relate to changes in motion.

This introduction will get you up to speed on the tools we are using to explore
dynamics: HuskyCT, Google Documents, Adams multibody dynamics, and Piazza
discussion board.


\section{Objectives}
\label{\detokenize{module_00/introduction:objectives}}
Upon completion of this module you will be able to
\begin{enumerate}
\sphinxsetlistlabels{\arabic}{enumi}{enumii}{}{.}%
\item {} 
Recognize the course objectives, requirements, grading policy, required
software, and materials

\item {} 
View video content and submit answers in a Google Form

\item {} 
Use the discussion board on Piazza

\item {} 
Locate course areas and features, and recognize their general purposes

\item {} 
Download and install Respondus Lockdown Browser with Monitor to ensure you
can successfully take the graded quizzes.

\item {} 
Scan and upload a hand\sphinxhyphen{}written assignment

\item {} 
Open an Adams file and save a figure

\item {} 
Submit a Google Document in HuskyCT

\end{enumerate}


\section{Activities}
\label{\detokenize{module_00/introduction:activities}}\begin{itemize}
\item {} 
Read the syllabus and post a discussion response on Piazza

\item {} 
Acquaint yourself with the course instructor.

\item {} 
Verify your computer settings are HuskyCT compatible.

\item {} 
Read about the course’s organization and tools.

\item {} 
Take a 5\sphinxhyphen{}minute practice quiz

\item {} 
Submit a practice quiz.

\item {} 
Download Respondus Lockdown Browser on the computer you plan to use in this
course.

\item {} 
Review the University of Connecticut’s academic policies.

\item {} 
Watch the Introduction videos and answer the questions

\item {} 
Finish a tutorial on Adams and complete the Google form

\item {} 
Turn in a 1\sphinxhyphen{}page report that has a figure from the Adams tutorial

\end{itemize}




\section{Participation Videos}
\label{\detokenize{module_00/participation:participation-videos}}\label{\detokenize{module_00/participation::doc}}\begin{enumerate}
\sphinxsetlistlabels{\arabic}{enumi}{enumii}{}{.}%
\item {} 
\sphinxhref{https://forms.gle/DDZMmb5J9iYPHfsu7}{Course Tour Video}\{target=”\_blank”\}

\item {} 
\sphinxhref{https://forms.gle/mzTody44zTP56PsL8}{Adams tour: rolling on incline}\{target=”\_blank”\}
\sphinxhref{https://drive.google.com/file/d/1wsRFfPXZm3axgzzLJFUh-eWwPBaLCeLk/view?usp=sharing}{rolling.bin example
file}\{target=”\_blank”\}

\end{enumerate}


\chapter{Module 1 \sphinxhyphen{} Dynamic equations and definitions}
\label{\detokenize{module_01/overview:module-1-dynamic-equations-and-definitions}}\label{\detokenize{module_01/overview::doc}}
\sphinxstyleemphasis{ball rolling on plane}

In this module, we want to understand what makes objects move and how we
can describe motion. In previous physics courses, you might remember the
\sphinxhref{https://phys.libretexts.org/Bookshelves/University\_Physics/Book\%3A\_University\_Physics\_(OpenStax)/Map\%3A\_University\_Physics\_I\_-\_Mechanics\_Sound\_Oscillations\_and\_Waves\_(OpenStax)/03\%3A\_Motion\_Along\_a\_Straight\_Line/3.07\%3A\_Free\_Fall}{kinematic
equations}
that describe the motion of a free\sphinxhyphen{}falling object.  How do these
equations relate to a ball rolling on the ground? or a domino falling
over?

As an engineer, you will need to design objects that move or interact
with moving objects. You need quantitative descriptions for impacts,
forces, and work. This module will give you the tools to describe motion,
forces, work, and impact.


\section{Outline}
\label{\detokenize{module_01/overview:outline}}\begin{enumerate}
\sphinxsetlistlabels{\arabic}{enumi}{enumii}{}{.}%
\item {} 
Kinematics
\begin{enumerate}
\sphinxsetlistlabels{\arabic}{enumii}{enumiii}{}{.}%
\item {} 
define position

\item {} 
velocity

\item {} 
acceleration

\item {} 
degrees of freedom and coordinates

\end{enumerate}

\item {} 
Kinetics
\begin{enumerate}
\sphinxsetlistlabels{\arabic}{enumii}{enumiii}{}{.}%
\item {} 
Newton\sphinxhyphen{}Euler equations

\item {} 
Impulse\sphinxhyphen{}momentum

\item {} 
Work\sphinxhyphen{}energy

\item {} 
equations of motion = second order diff eq

\end{enumerate}

\item {} 
Solving dynamic problem
\begin{enumerate}
\sphinxsetlistlabels{\arabic}{enumii}{enumiii}{}{.}%
\item {} 
Newton\sphinxhyphen{}Euler equations define eom

\item {} 
Use ODE solution for x(t)

\end{enumerate}

\end{enumerate}


\subsection{Geometry of motion \sphinxhyphen{} kinematics}
\label{\detokenize{module_01/kinematics:geometry-of-motion-kinematics}}\label{\detokenize{module_01/kinematics::doc}}

\subsubsection{Position}
\label{\detokenize{module_01/kinematics:position}}
Classical physics describes the position of an object using three
independent coordinates e.g.


\label{equation:module_01/kinematics:a54917d7-7520-45f3-9ae3-c9648fcb0ed4}\begin{equation}
\mathbf{r}_{P/O} = x\hat{i} + y\hat{j} + z\hat{k}
\end{equation}
where \(\mathbf{r}_{P/O}\) is the position of point \(P\) \sphinxstyleemphasis{with respect to the point
of origin} \(O\), \(x,~y,~z\) are magnitudes of distance along a
Cartesian coordinate system and \(\hat{i},~\hat{j}\) and \(\hat{k}\) are
unit vectors that describe three

\begin{sphinxVerbatim}[commandchars=\\\{\}]
\PYG{k+kn}{import} \PYG{n+nn}{numpy} \PYG{k}{as} \PYG{n+nn}{np}
\PYG{k+kn}{import} \PYG{n+nn}{matplotlib}\PYG{n+nn}{.}\PYG{n+nn}{pyplot} \PYG{k}{as} \PYG{n+nn}{plt}
\PYG{n}{plt}\PYG{o}{.}\PYG{n}{style}\PYG{o}{.}\PYG{n}{use}\PYG{p}{(}\PYG{l+s+s1}{\PYGZsq{}}\PYG{l+s+s1}{fivethirtyeight}\PYG{l+s+s1}{\PYGZsq{}}\PYG{p}{)}
\end{sphinxVerbatim}


\subsubsection{Velocity}
\label{\detokenize{module_01/kinematics:velocity}}
The velocity of an object is the change in {\hyperref[\detokenize{module_01/kinematics:position}]{\emph{position}}} per length of time.

\label{equation:module_01/kinematics:3c98d8a4-e9bf-4aba-acab-99b7d3660a3a}\begin{equation}
\mathbf{v}_{P/O} = \frac{d\mathbf{r}_{P/O}}{dt} = \dot{x}\hat{i} + \dot{y}\hat{j} +
\dot{z}\hat{k}
\end{equation}\begin{quote}

\sphinxstylestrong{Note:} The notation \(\dot{x}\) and \(\ddot{x}\) is short\sphinxhyphen{}hand for writing out
\(\frac{dx}{dt}\) and \(\frac{d^2x}{dt^2}\), respectively.
\end{quote}

The definition of velocity in equation \eqref{equation:kinematics:velocity} depends upon the change in position of all
three independent coordinates, where
\(\frac{d}{dt}(x\hat{i})=\dot{x}\hat{i}\).
\begin{quote}

\sphinxstylestrong{Note:} Remember the chain rule:
\(\frac{d}{dt}(x\hat{i})=\dot{x}\hat{i} +
x\dot{\hat{i}}\), but \(\dot{\hat{i}}=0\) because this unit vector is not
changing direction. You’ll see other unit vectors later that do change.
\end{quote}


\subsubsection{Example \sphinxhyphen{} GPS vs speedometer}
\label{\detokenize{module_01/kinematics:example-gps-vs-speedometer}}
You can find velocity based upon postion, but you can only find changes
in position with velocity. Consider tracking the motion of a car driving
down a road using GPS. You determine its motion and create the position,
\(\mathbf{r} = x\hat{i} +y\hat{j}\), where

\(x(t) = 4t +3\) and \(y(t) = 3t - 1\)

To get the velocity, calculate \(\mathbf{v} = \dot{\mathbf{r}}\)

\(\mathbf{v} = 4\hat{i} +3 \hat{j}\)

\begin{sphinxVerbatim}[commandchars=\\\{\}]
\PYG{n}{t} \PYG{o}{=} \PYG{n}{np}\PYG{o}{.}\PYG{n}{arange}\PYG{p}{(}\PYG{l+m+mi}{0}\PYG{p}{,}\PYG{l+m+mi}{5}\PYG{p}{)}
\PYG{n}{x} \PYG{o}{=} \PYG{l+m+mi}{4}\PYG{o}{*}\PYG{n}{t} \PYG{o}{+} \PYG{l+m+mi}{3}
\PYG{n}{y} \PYG{o}{=} \PYG{l+m+mi}{3}\PYG{o}{*}\PYG{n}{t} \PYG{o}{\PYGZhy{}}\PYG{l+m+mi}{1}
\PYG{n}{plt}\PYG{o}{.}\PYG{n}{plot}\PYG{p}{(}\PYG{n}{x}\PYG{p}{,}\PYG{n}{y}\PYG{p}{,}\PYG{l+s+s1}{\PYGZsq{}}\PYG{l+s+s1}{o}\PYG{l+s+s1}{\PYGZsq{}}\PYG{p}{)}
\PYG{n}{plt}\PYG{o}{.}\PYG{n}{quiver}\PYG{p}{(}\PYG{n}{x}\PYG{p}{,}\PYG{n}{y}\PYG{p}{,}\PYG{l+m+mi}{4}\PYG{p}{,}\PYG{l+m+mi}{3}\PYG{p}{)}
\PYG{n}{plt}\PYG{o}{.}\PYG{n}{title}\PYG{p}{(}\PYG{l+s+s1}{\PYGZsq{}}\PYG{l+s+s1}{Position of car on road every 1 second}\PYG{l+s+s1}{\PYGZsq{}}\PYG{o}{+}
    \PYG{l+s+s1}{\PYGZsq{}}\PYG{l+s+se}{\PYGZbs{}n}\PYG{l+s+s1}{velocity shown as arrow}\PYG{l+s+s1}{\PYGZsq{}}\PYG{p}{)}
\PYG{n}{plt}\PYG{o}{.}\PYG{n}{xlabel}\PYG{p}{(}\PYG{l+s+s1}{\PYGZsq{}}\PYG{l+s+s1}{x\PYGZhy{}position (m)}\PYG{l+s+s1}{\PYGZsq{}}\PYG{p}{)}
\PYG{n}{plt}\PYG{o}{.}\PYG{n}{ylabel}\PYG{p}{(}\PYG{l+s+s1}{\PYGZsq{}}\PYG{l+s+s1}{y\PYGZhy{}position (m)}\PYG{l+s+s1}{\PYGZsq{}}\PYG{p}{)}\PYG{p}{;}
\end{sphinxVerbatim}

\noindent\sphinxincludegraphics{{kinematics_5_0}.png}


\subsubsection{Speed}
\label{\detokenize{module_01/kinematics:speed}}
The speed of an object is the
\sphinxhref{https://www.mathsisfun.com/algebra/vectors.html}{magnitude} of the
velocity,

\(|\mathbf{v}_{P/O}| = \sqrt{\mathbf{v}\cdot\mathbf{v}} =
\sqrt{\dot{x}^2 + \dot{y}^2 + \dot{z}^2}\)


\subsubsection{Acceleration}
\label{\detokenize{module_01/kinematics:acceleration}}
The acceleration of an object is the change in velocity per length of
time.

\(\mathbf{a}_{P/O} = \frac{d \mathbf{v}_{P/O} }{dt} = \ddot{x}\hat{i} +
\ddot{y}\hat{j} + \ddot{z}\hat{k}\)

where \(\ddot{x}=\frac{d^2 x}{dt^2}\) and \(\mathbf{a}_{P/O}\) is the
acceleration of point \(P\) \sphinxstyleemphasis{with respect to the point of origin} \(O\).


\subsubsection{Rotation and Orientation}
\label{\detokenize{module_01/kinematics:rotation-and-orientation}}
The definitions of position, velocity, and acceleration all describe a
single point, but dynamic engineering systems are composed of rigid
bodies is needed to describe the position of an object.

\begin{sphinxVerbatim}[commandchars=\\\{\}]
\PYG{k+kn}{from} \PYG{n+nn}{IPython}\PYG{n+nn}{.}\PYG{n+nn}{core}\PYG{n+nn}{.}\PYG{n+nn}{display} \PYG{k+kn}{import} \PYG{n}{SVG}

\PYG{n}{SVG}\PYG{p}{(}\PYG{n}{filename}\PYG{o}{=}\PYG{l+s+s1}{\PYGZsq{}}\PYG{l+s+s1}{./images/position\PYGZus{}angle.svg}\PYG{l+s+s1}{\PYGZsq{}}\PYG{p}{)}
\end{sphinxVerbatim}

\begin{sphinxVerbatim}[commandchars=\\\{\}]
\PYGZlt{}IPython.core.display.SVG object\PYGZgt{}
\end{sphinxVerbatim}

\sphinxstyleemphasis{In the figure above, the center of the block is located at
\(r_{P/O}=x\hat{i}+y\hat{j}\) in both the left and right images, but the
two locations are not the same. The orientation of the block is
important for determining the position of all the material points.}

In general, a rigid body has a \sphinxstyleemphasis{pitch}, \sphinxstyleemphasis{yaw}, and \sphinxstyleemphasis{roll} that describes
its rotational orientation, as seen in the animation below. We will
revisit 3D motion in Module\_05

\begin{sphinxVerbatim}[commandchars=\\\{\}]
\PYG{k+kn}{from} \PYG{n+nn}{IPython}\PYG{n+nn}{.}\PYG{n+nn}{display} \PYG{k+kn}{import} \PYG{n}{YouTubeVideo}
\PYG{n}{vid} \PYG{o}{=} \PYG{n}{YouTubeVideo}\PYG{p}{(}\PYG{l+s+s2}{\PYGZdq{}}\PYG{l+s+s2}{li7t\PYGZhy{}\PYGZhy{}8UZms?loop=1}\PYG{l+s+s2}{\PYGZdq{}}\PYG{p}{)}
\PYG{n}{display}\PYG{p}{(}\PYG{n}{vid}\PYG{p}{)}
\end{sphinxVerbatim}

\begin{sphinxVerbatim}[commandchars=\\\{\}]
\PYGZlt{}IPython.lib.display.YouTubeVideo at 0x7f4888ac82b0\PYGZgt{}
\end{sphinxVerbatim}


\subsubsection{Rotation in planar motion}
\label{\detokenize{module_01/kinematics:rotation-in-planar-motion}}
Our initial focus is planar rotations e.g. yaw and roll are fixed. For a
body constrained to planar motion, you need 3 independent measurements
to describe its position e.g. \(x\), \(y\), and \(\theta\)

\begin{sphinxVerbatim}[commandchars=\\\{\}]
\PYG{k+kn}{import} \PYG{n+nn}{numpy} \PYG{k}{as} \PYG{n+nn}{np}
\PYG{k+kn}{import} \PYG{n+nn}{matplotlib}\PYG{n+nn}{.}\PYG{n+nn}{pyplot} \PYG{k}{as} \PYG{n+nn}{plt}
\PYG{n}{plt}\PYG{o}{.}\PYG{n}{style}\PYG{o}{.}\PYG{n}{use}\PYG{p}{(}\PYG{l+s+s1}{\PYGZsq{}}\PYG{l+s+s1}{fivethirtyeight}\PYG{l+s+s1}{\PYGZsq{}}\PYG{p}{)}
\PYG{k+kn}{from} \PYG{n+nn}{IPython}\PYG{n+nn}{.}\PYG{n+nn}{core}\PYG{n+nn}{.}\PYG{n+nn}{display} \PYG{k+kn}{import} \PYG{n}{SVG}
\end{sphinxVerbatim}


\subsection{Mechanical advantage}
\label{\detokenize{module_01/mechanical-advantage:mechanical-advantage}}\label{\detokenize{module_01/mechanical-advantage::doc}}
In the diagram below, there is a 50\sphinxhyphen{}N force, \(F\), applied downwards on a
constrained linkage system. There is a restoring force, \(R\), that
maintains a slow change in the angle of the mechanism, \(\theta\), as it
changes from \(89^o\) to \(-89^o\). Your
goal is to determine the necessary restoring force \(R\) as a
function of angle \(\theta\).

\sphinxincludegraphics{{two-bar}.svg}


\subsubsection{Kinematics \sphinxhyphen{} geometry of motion}
\label{\detokenize{module_01/mechanical-advantage:kinematics-geometry-of-motion}}
This system has two rigid bodies connected at point \(A\) by a pin and the
whole system is pinned to the ground at \(O\). Point \(B\) slides
along a horizontal surface. The total degrees of freedom are

3 (link 1) + 3 (link 2) \sphinxhyphen{} 2 constraints (\(O\)) \sphinxhyphen{} 2 constraints (\(A\)) \sphinxhyphen{} 1
constraint (\(B\)) \sphinxhyphen{} 1 constraint (\(\theta\)\sphinxhyphen{}fixed) = \sphinxstylestrong{0 DOF}

For a given angle \(\theta\), there can only be one configuration in the
system. Create the constraint equations using the relative position of
point \(A\) as such

\(\mathbf{r}_A =  \mathbf{r}_{A/B} + \mathbf{r}_B\)

\sphinxincludegraphics{{two-bar_constraints}.svg}

where

\(\mathbf{r}_A = L(\cos\theta \hat{i}+ \sin\theta \hat{j})\)

\(\mathbf{r}_B = d\hat{i}\)

\(\mathbf{r}_{A/B} = L(-\cos\theta \hat{i} + \sin\theta \hat{j})\)

solving for the \(\hat{i}-\) and \(\hat{j}-\)components creates two
equations that describe the state of the system
\begin{itemize}
\item {} 
\(\theta = \theta\) which states because we have an isosceles triangle,
two angles have to be equal

\item {} 
\(d = 2 L \cos\theta\) so \(\mathbf{r}_{B} = 2L \cos\theta \hat{i}\)

\end{itemize}

\begin{sphinxVerbatim}[commandchars=\\\{\}]
\PYG{n}{theta} \PYG{o}{=} \PYG{n}{np}\PYG{o}{.}\PYG{n}{linspace}\PYG{p}{(}\PYG{l+m+mi}{89}\PYG{p}{,}\PYG{o}{\PYGZhy{}}\PYG{l+m+mi}{89}\PYG{p}{)}\PYG{o}{*}\PYG{n}{np}\PYG{o}{.}\PYG{n}{pi}\PYG{o}{/}\PYG{l+m+mi}{180}
\PYG{n}{d} \PYG{o}{=} \PYG{l+m+mi}{2}\PYG{o}{*}\PYG{l+m+mi}{1}\PYG{o}{*}\PYG{n}{np}\PYG{o}{.}\PYG{n}{cos}\PYG{p}{(}\PYG{n}{theta}\PYG{p}{)}
\PYG{n}{plt}\PYG{o}{.}\PYG{n}{plot}\PYG{p}{(}\PYG{n}{theta}\PYG{o}{*}\PYG{l+m+mi}{180}\PYG{o}{/}\PYG{n}{np}\PYG{o}{.}\PYG{n}{pi}\PYG{p}{,} \PYG{n}{d}\PYG{p}{)}
\PYG{n}{plt}\PYG{o}{.}\PYG{n}{xticks}\PYG{p}{(}\PYG{n}{np}\PYG{o}{.}\PYG{n}{arange}\PYG{p}{(}\PYG{o}{\PYGZhy{}}\PYG{l+m+mi}{90}\PYG{p}{,}\PYG{l+m+mi}{91}\PYG{p}{,}\PYG{l+m+mi}{30}\PYG{p}{)}\PYG{p}{)}
\PYG{n}{plt}\PYG{o}{.}\PYG{n}{xlabel}\PYG{p}{(}\PYG{l+s+sa}{r}\PYG{l+s+s1}{\PYGZsq{}}\PYG{l+s+s1}{\PYGZdl{}}\PYG{l+s+s1}{\PYGZbs{}}\PYG{l+s+s1}{theta\PYGZdl{} (deg)}\PYG{l+s+s1}{\PYGZsq{}}\PYG{p}{)}
\PYG{n}{plt}\PYG{o}{.}\PYG{n}{xlim}\PYG{p}{(}\PYG{l+m+mi}{90}\PYG{p}{,}\PYG{o}{\PYGZhy{}}\PYG{l+m+mi}{90}\PYG{p}{)}
\PYG{n}{plt}\PYG{o}{.}\PYG{n}{ylabel}\PYG{p}{(}\PYG{l+s+s1}{\PYGZsq{}}\PYG{l+s+s1}{d (m)}\PYG{l+s+s1}{\PYGZsq{}}\PYG{p}{)}\PYG{p}{;}
\end{sphinxVerbatim}

\noindent\sphinxincludegraphics{{mechanical-advantage_3_0}.png}


\subsubsection{Kinetics \sphinxhyphen{} applied forces and constraint forces}
\label{\detokenize{module_01/mechanical-advantage:kinetics-applied-forces-and-constraint-forces}}
The applied force, \(F=50~N\) is constant, but \(R\) is dependent upon the
geometry of the system. You solved for the kinematics in the first part,
here you can use the Newton\sphinxhyphen{}Euler equations to solve for \(R\) given \(\theta\) and
\(F\). Separate the system into the left and right links,

\sphinxincludegraphics{{two-bar_FBD}.svg}

The Newton\sphinxhyphen{}Euler equations:
\begin{itemize}
\item {} 
\(\mathbf{F} = m\mathbf{a} = \mathbf{0}\) links moving slowly

\item {} 
\(M_G = I\alpha = 0\) links rotating slowly

\end{itemize}

Newton\sphinxhyphen{}Euler equations for the left bar:
\begin{enumerate}
\sphinxsetlistlabels{\arabic}{enumi}{enumii}{}{.}%
\item {} 
\(\mathbf{F}\cdot \hat{i} = N_{x1}+N_{x2} = 0\)

\item {} 
\(\mathbf{F}\cdot \hat{i} = N_{y1}+N_{y2} - F = 0\)

\item {} 
\(M_O = l\hat{b}_1 \times (-F\hat{j} + N_{x2}\hat{i} + N_{y2}\hat{j})
= 0\)

\end{enumerate}

Newton\sphinxhyphen{}Euler equations for the right bar:
\begin{enumerate}
\sphinxsetlistlabels{\arabic}{enumi}{enumii}{}{.}%
\item {} 
\(\mathbf{F}\cdot \hat{i} = -N_{x2}-R = 0\)

\item {} 
\(\mathbf{F}\cdot \hat{i} = -N_{y2}+N_{y3} = 0\)

\item {} 
\(M_A = l\hat{c}_1 \times (-R\hat{i} + N_{y3}\hat{j})= 0\)

\end{enumerate}
\begin{quote}

\sphinxstylestrong{Note:} Don’t count \(F\) twice! You can use the applied force on the
left or right, but not both. Try solving the equations placing it on
the right bar.
\end{quote}

The four equations for \(\mathbf{F}\) relate the reaction forces
\(\mathbf{N}_{1},~\mathbf{N}_{2},~and~\mathbf{N}_{3}\). The two moment
equations relate \(R\) to \(F\) and \(\theta\) as such

\(N_{y2}L\cos\theta - RL\sin\theta = 0\)

and

\(-FL\cos\theta +N_{y2}\cos\theta+RL\sin\theta = 0\)

combining there results

\(F\cos\theta = 2R\sin\theta\rightarrow R = \frac{F}{2}\cot\theta\)

\begin{sphinxVerbatim}[commandchars=\\\{\}]
\PYG{n}{R} \PYG{o}{=} \PYG{l+m+mi}{50}\PYG{o}{/}\PYG{l+m+mi}{2}\PYG{o}{*}\PYG{n}{np}\PYG{o}{.}\PYG{n}{tan}\PYG{p}{(}\PYG{n}{theta}\PYG{p}{)}\PYG{o}{*}\PYG{o}{*}\PYG{o}{\PYGZhy{}}\PYG{l+m+mi}{1}
\PYG{n}{plt}\PYG{o}{.}\PYG{n}{plot}\PYG{p}{(}\PYG{n}{theta}\PYG{o}{*}\PYG{l+m+mi}{180}\PYG{o}{/}\PYG{n}{np}\PYG{o}{.}\PYG{n}{pi}\PYG{p}{,} \PYG{n}{R}\PYG{p}{)}
\PYG{n}{plt}\PYG{o}{.}\PYG{n}{xticks}\PYG{p}{(}\PYG{n}{np}\PYG{o}{.}\PYG{n}{arange}\PYG{p}{(}\PYG{o}{\PYGZhy{}}\PYG{l+m+mi}{90}\PYG{p}{,}\PYG{l+m+mi}{91}\PYG{p}{,}\PYG{l+m+mi}{30}\PYG{p}{)}\PYG{p}{)}
\PYG{n}{plt}\PYG{o}{.}\PYG{n}{xlabel}\PYG{p}{(}\PYG{l+s+sa}{r}\PYG{l+s+s1}{\PYGZsq{}}\PYG{l+s+s1}{\PYGZdl{}}\PYG{l+s+s1}{\PYGZbs{}}\PYG{l+s+s1}{theta\PYGZdl{} (deg)}\PYG{l+s+s1}{\PYGZsq{}}\PYG{p}{)}
\PYG{n}{plt}\PYG{o}{.}\PYG{n}{ylabel}\PYG{p}{(}\PYG{l+s+s1}{\PYGZsq{}}\PYG{l+s+s1}{restoring force, R (N)}\PYG{l+s+s1}{\PYGZsq{}}\PYG{p}{)}
\PYG{n}{plt}\PYG{o}{.}\PYG{n}{xlim}\PYG{p}{(}\PYG{l+m+mi}{90}\PYG{p}{,}\PYG{o}{\PYGZhy{}}\PYG{l+m+mi}{90}\PYG{p}{)}
\PYG{n}{plt}\PYG{o}{.}\PYG{n}{ylim}\PYG{p}{(}\PYG{o}{\PYGZhy{}}\PYG{l+m+mi}{200}\PYG{p}{,}\PYG{l+m+mi}{200}\PYG{p}{)}\PYG{p}{;}
\end{sphinxVerbatim}

\noindent\sphinxincludegraphics{{mechanical-advantage_7_0}.png}


\subsubsection{Wrapping up}
\label{\detokenize{module_01/mechanical-advantage:wrapping-up}}
Take a look at the mechanical advantage and disadvantage this system can create. For angles close to \(\theta\approx 90^o\), the restoring force is close to zero. In this case, the applied force is mostly directed at constraints on the system. When the angles are close to \(\theta \approx 0^o\), the required restoring force can be \(>100\times\) the input force.

Have you seen this type of linkage system in engineering devices?

\begin{sphinxVerbatim}[commandchars=\\\{\}]
\PYG{k+kn}{import} \PYG{n+nn}{numpy} \PYG{k}{as} \PYG{n+nn}{np}
\PYG{k+kn}{import} \PYG{n+nn}{matplotlib}\PYG{n+nn}{.}\PYG{n+nn}{pyplot} \PYG{k}{as} \PYG{n+nn}{plt}
\PYG{n}{plt}\PYG{o}{.}\PYG{n}{style}\PYG{o}{.}\PYG{n}{use}\PYG{p}{(}\PYG{l+s+s1}{\PYGZsq{}}\PYG{l+s+s1}{fivethirtyeight}\PYG{l+s+s1}{\PYGZsq{}}\PYG{p}{)}
\PYG{n}{plt}\PYG{o}{.}\PYG{n}{rcParams}\PYG{o}{.}\PYG{n}{update}\PYG{p}{(}\PYG{p}{\PYGZob{}}
  \PYG{l+s+s2}{\PYGZdq{}}\PYG{l+s+s2}{text.usetex}\PYG{l+s+s2}{\PYGZdq{}}\PYG{p}{:} \PYG{k+kc}{True}\PYG{p}{,}
  \PYG{l+s+s2}{\PYGZdq{}}\PYG{l+s+s2}{font.sans\PYGZhy{}serif}\PYG{l+s+s2}{\PYGZdq{}}\PYG{p}{:} \PYG{p}{[}\PYG{l+s+s2}{\PYGZdq{}}\PYG{l+s+s2}{Helvetica}\PYG{l+s+s2}{\PYGZdq{}}\PYG{p}{]}\PYG{p}{\PYGZcb{}}\PYG{p}{)}
\end{sphinxVerbatim}


\subsection{Driving forces for moving systems}
\label{\detokenize{module_01/driving-forces:driving-forces-for-moving-systems}}\label{\detokenize{module_01/driving-forces::doc}}
In this case study, you want to accelerate a 0.1\sphinxhyphen{}kg flywheel with a
piston. The desired
acceleration of the flywheel is \(\alpha=50~rad/s^2.\) The piston is
attached to the link and creates a horizontal driving force. This
example demonstrates how we can describe constraints in mathematical
forms.

We have two moving bodies:
\begin{enumerate}
\sphinxsetlistlabels{\arabic}{enumi}{enumii}{}{.}%
\item {} 
a flywheel with radius \(r=0.2~m\) and mass \(m=0.1~kg\)

\item {} 
a massless connecting link

\end{enumerate}

Both objects move in a horizontal plane, so their positions have two
degrees of freedom to describe position and one degree of freedom that
describes orientation.

\(DOF = 3 + 3 = 6\)

\begin{sphinxVerbatim}[commandchars=\\\{\}]
\PYG{k+kn}{from} \PYG{n+nn}{IPython}\PYG{n+nn}{.}\PYG{n+nn}{core}\PYG{n+nn}{.}\PYG{n+nn}{display} \PYG{k+kn}{import} \PYG{n}{SVG}

\PYG{n}{SVG}\PYG{p}{(}\PYG{n}{filename}\PYG{o}{=}\PYG{l+s+s1}{\PYGZsq{}}\PYG{l+s+s1}{./images/piston\PYGZhy{}flywheel.svg}\PYG{l+s+s1}{\PYGZsq{}}\PYG{p}{)}
\end{sphinxVerbatim}

\begin{sphinxVerbatim}[commandchars=\\\{\}]
\PYGZlt{}IPython.core.display.SVG object\PYGZgt{}
\end{sphinxVerbatim}


\subsubsection{Describing constraints}
\label{\detokenize{module_01/driving-forces:describing-constraints}}
There are six degrees of freedom for a flywheel and a link, but there
are 6 constraints on the system’s motion:
\begin{enumerate}
\sphinxsetlistlabels{\arabic}{enumi}{enumii}{}{.}%
\item {} 
the flywheel is pinned to the ground in the x\sphinxhyphen{}dir

\item {} 
the flywheel is pinned to the ground in the y\sphinxhyphen{}dir

\item {} 
the top of the link is pinned to the flywheel

\item {} 
the top of the link is pinned to the flywheel

\item {} 
the bottom of the link slides along the horizontal line

\item {} 
the angle of the flywheel has acceleration, \(\alpha=50~rad/s^2\)

\end{enumerate}

\(system~DOF = 6 - 6 = 0~DOF\)
\begin{quote}

\sphinxstylestrong{Note:} In general, a pin in a planar system creates 2 constraints on motion.
\end{quote}

You should recognize at this point that there are \sphinxstyleemphasis{0 differential
equations to solve}. Once you have calculated the kinematics based upon
the system constraints, you can plug in the values and solve for force
of the piston. So, start with the kinematic description of motion.

\(\mathbf{r}_2 =\mathbf{r}_{2/3} + \mathbf{r}_3\)

\begin{sphinxVerbatim}[commandchars=\\\{\}]
\PYG{n}{SVG}\PYG{p}{(}\PYG{n}{filename} \PYG{o}{=} \PYG{l+s+s1}{\PYGZsq{}}\PYG{l+s+s1}{./images/flywheel\PYGZhy{}constraints.svg}\PYG{l+s+s1}{\PYGZsq{}}\PYG{p}{)}
\end{sphinxVerbatim}

\begin{sphinxVerbatim}[commandchars=\\\{\}]
\PYGZlt{}IPython.core.display.SVG object\PYGZgt{}
\end{sphinxVerbatim}

\(r(\sin\theta\hat{i} - \cos\theta \hat{j}) =
-L(\cos\theta_{L}\hat{i}-\sin\theta_L\hat{j}) + d\hat{i}\)

this description creates two independent equations
\begin{enumerate}
\sphinxsetlistlabels{\arabic}{enumi}{enumii}{}{.}%
\item {} 
\(-r\sin\theta = -L\cos\theta_{L}+ d\)

\item {} 
\(r\cos\theta = -L\sin\theta_L\)

\end{enumerate}

The constraint on \(\theta\) says that \(\theta(t)=\frac{\alpha t^2}{2}\).
You need to solve for \(\theta_L\) and \(d\) to determine the full
state of the system.

\(\theta_L = \sin^{-1}\left(\frac{r}{L}\cos\theta\right)\)

\(d = L\cos\theta_{L}-r\sin\theta\)

\begin{sphinxVerbatim}[commandchars=\\\{\}]
\PYG{n}{r} \PYG{o}{=} \PYG{l+m+mf}{0.2}
\PYG{n}{L} \PYG{o}{=} \PYG{l+m+mf}{0.5}
\PYG{n}{theta} \PYG{o}{=} \PYG{n}{np}\PYG{o}{.}\PYG{n}{linspace}\PYG{p}{(}\PYG{l+m+mi}{0}\PYG{p}{,}\PYG{l+m+mi}{2}\PYG{o}{*}\PYG{n}{np}\PYG{o}{.}\PYG{n}{pi}\PYG{p}{,}\PYG{l+m+mi}{100}\PYG{p}{)}
\PYG{n}{thetaL} \PYG{o}{=} \PYG{n}{np}\PYG{o}{.}\PYG{n}{arcsin}\PYG{p}{(}\PYG{n}{r}\PYG{o}{/}\PYG{n}{L}\PYG{o}{*}\PYG{n}{np}\PYG{o}{.}\PYG{n}{cos}\PYG{p}{(}\PYG{n}{theta}\PYG{p}{)}\PYG{p}{)}
\PYG{n}{d} \PYG{o}{=} \PYG{n}{L}\PYG{o}{*}\PYG{n}{np}\PYG{o}{.}\PYG{n}{cos}\PYG{p}{(}\PYG{n}{thetaL}\PYG{p}{)}\PYG{o}{\PYGZhy{}}\PYG{n}{r}\PYG{o}{*}\PYG{n}{np}\PYG{o}{.}\PYG{n}{sin}\PYG{p}{(}\PYG{n}{theta}\PYG{p}{)}
\PYG{n}{f}\PYG{p}{,} \PYG{n}{ax} \PYG{o}{=} \PYG{n}{plt}\PYG{o}{.}\PYG{n}{subplots}\PYG{p}{(}\PYG{p}{)}
\PYG{n}{ax}\PYG{o}{.}\PYG{n}{plot}\PYG{p}{(}\PYG{n}{theta}\PYG{o}{*}\PYG{l+m+mi}{180}\PYG{o}{/}\PYG{n}{np}\PYG{o}{.}\PYG{n}{pi}\PYG{p}{,} \PYG{n}{thetaL}\PYG{o}{*}\PYG{l+m+mi}{180}\PYG{o}{/}\PYG{n}{np}\PYG{o}{.}\PYG{n}{pi}\PYG{p}{,}\PYG{l+s+s1}{\PYGZsq{}}\PYG{l+s+s1}{b\PYGZhy{}}\PYG{l+s+s1}{\PYGZsq{}}\PYG{p}{,}\PYG{n}{label} \PYG{o}{=} \PYG{l+s+sa}{r}\PYG{l+s+s1}{\PYGZsq{}}\PYG{l+s+s1}{\PYGZbs{}}\PYG{l+s+s1}{theta\PYGZus{}L}\PYG{l+s+s1}{\PYGZsq{}}\PYG{p}{)}
\PYG{n}{ax}\PYG{o}{.}\PYG{n}{set\PYGZus{}xlabel}\PYG{p}{(}\PYG{l+s+sa}{r}\PYG{l+s+s1}{\PYGZsq{}}\PYG{l+s+s1}{\PYGZdl{}}\PYG{l+s+s1}{\PYGZbs{}}\PYG{l+s+s1}{theta\PYGZdl{} (degrees)}\PYG{l+s+s1}{\PYGZsq{}}\PYG{p}{)}
\PYG{n}{ax}\PYG{o}{.}\PYG{n}{set\PYGZus{}ylabel}\PYG{p}{(}\PYG{l+s+sa}{r}\PYG{l+s+s1}{\PYGZsq{}}\PYG{l+s+s1}{\PYGZdl{}}\PYG{l+s+s1}{\PYGZbs{}}\PYG{l+s+s1}{theta\PYGZus{}L\PYGZdl{} (degrees)}\PYG{l+s+s1}{\PYGZsq{}}\PYG{p}{,}\PYG{n}{color} \PYG{o}{=} \PYG{l+s+s1}{\PYGZsq{}}\PYG{l+s+s1}{blue}\PYG{l+s+s1}{\PYGZsq{}}\PYG{p}{)}
\PYG{n}{plt}\PYG{o}{.}\PYG{n}{xticks}\PYG{p}{(}\PYG{n}{np}\PYG{o}{.}\PYG{n}{arange}\PYG{p}{(}\PYG{l+m+mi}{0}\PYG{p}{,}\PYG{l+m+mi}{360}\PYG{p}{,}\PYG{l+m+mi}{30}\PYG{p}{)}\PYG{p}{)}
\PYG{n}{ax2} \PYG{o}{=} \PYG{n}{ax}\PYG{o}{.}\PYG{n}{twinx}\PYG{p}{(}\PYG{p}{)}
\PYG{n}{ax2}\PYG{o}{.}\PYG{n}{plot}\PYG{p}{(}\PYG{n}{theta}\PYG{o}{*}\PYG{l+m+mi}{180}\PYG{o}{/}\PYG{n}{np}\PYG{o}{.}\PYG{n}{pi}\PYG{p}{,} \PYG{n}{d}\PYG{p}{,} \PYG{l+s+s1}{\PYGZsq{}}\PYG{l+s+s1}{g\PYGZhy{}\PYGZhy{}}\PYG{l+s+s1}{\PYGZsq{}}\PYG{p}{,} \PYG{n}{label} \PYG{o}{=} \PYG{l+s+s1}{\PYGZsq{}}\PYG{l+s+s1}{piston d}\PYG{l+s+s1}{\PYGZsq{}}\PYG{p}{)}
\PYG{n}{ax2}\PYG{o}{.}\PYG{n}{set\PYGZus{}ylabel}\PYG{p}{(}\PYG{l+s+s1}{\PYGZsq{}}\PYG{l+s+s1}{d (meter)}\PYG{l+s+s1}{\PYGZsq{}}\PYG{p}{,} \PYG{n}{color} \PYG{o}{=} \PYG{l+s+s1}{\PYGZsq{}}\PYG{l+s+s1}{green}\PYG{l+s+s1}{\PYGZsq{}}\PYG{p}{)}\PYG{p}{;}
\PYG{n}{plt}\PYG{o}{.}\PYG{n}{show}\PYG{p}{(}\PYG{p}{)}
\end{sphinxVerbatim}

\noindent\sphinxincludegraphics{{driving-forces_6_0}.png}


\bigskip\hrule\bigskip



\paragraph{Exercise:}
\label{\detokenize{module_01/driving-forces:exercise}}
What is the angular velocity and angular acceleration of the link?
\(\dot{\theta}_L\) and \(\ddot{\theta}_L\)


\bigskip\hrule\bigskip


Now, we have solved all of the kinematics we need to
solve for the applied piston force. Sometimes, you can
look at the Newton\sphinxhyphen{}Euler equations for the syatem as a
whole, but here we need to separate each component and
sum forces and moments.

Flywheel:

\(\sum\mathbf{F}=\mathbf{0}\)

\(\sum M_G = I \alpha\)
\begin{quote}

\sphinxstylestrong{Note:} If a body is moving, you either have to sum moments about a fixed point or its center of mass. Otherwise, the description of angular acceleration is more involved.
\end{quote}

Link:

\(\sum\mathbf{F}=\mathbf{0}\)

\(\sum M = 0\)
\begin{quote}

\sphinxstylestrong{Note:} when links, cables, or pulleys are assumed to
be massless they still obey Newtonian mechanics, but
momentum is always zero. So the sum of forces or
moments is equal to 0.
\end{quote}

The three kinetic equations for the wheel are as such,
\begin{enumerate}
\sphinxsetlistlabels{\arabic}{enumi}{enumii}{}{.}%
\item {} 
\(\sum \mathbf{F}\cdot\hat{i} = F_{2x}+F_{1x} = 0\)

\item {} 
\(\sum \mathbf{F}\cdot\hat{j} = F_{2y}+F_{1y} = 0\)

\item {} 
\(\sum M_{1} = r\hat{e}_r \times (F_{2x}\hat{i}+F_{2y}\hat{j}) = \frac{mr^2}{2}\alpha\)

\end{enumerate}

and the three kinetic equations for the link are as such,
\begin{enumerate}
\sphinxsetlistlabels{\arabic}{enumi}{enumii}{}{.}%
\item {} 
\(\sum \mathbf{F}\cdot\hat{i} = R-F_{2x} = 0\)

\item {} 
\(\sum \mathbf{F}\cdot\hat{j} = -F_{2y} - F_{piston} = 0\)

\item {} 
\(\sum M_{2} = L\hat{b}_1 \times (-F_{piston}\hat{i} + R\hat{j}) = 0\)

\end{enumerate}

The third equation for the link is rearranged to solve for \(R =
F\tan\theta_L\). The third equation for the flywheel is rearranged to
solve for \(F\) as such

\(rF\cos\theta -rF\tan\theta_L\sin\theta = \frac{mr^2}{2}\alpha\)

finally arriving at

\(F = \frac{mr\alpha}{2}\left(\cos\theta-\tan\theta_L\sin\theta
\right)^{-1}\)

Plotted below as a function of flywheel angle, \(\theta\),

\begin{sphinxVerbatim}[commandchars=\\\{\}]
\PYG{n}{m} \PYG{o}{=} \PYG{l+m+mf}{0.1}
\PYG{n}{F} \PYG{o}{=} \PYG{n}{m}\PYG{o}{*}\PYG{n}{r}\PYG{o}{/}\PYG{l+m+mi}{2}\PYG{o}{*}\PYG{p}{(}\PYG{n}{np}\PYG{o}{.}\PYG{n}{cos}\PYG{p}{(}\PYG{n}{theta}\PYG{p}{)}\PYG{o}{\PYGZhy{}}\PYG{n}{np}\PYG{o}{.}\PYG{n}{tan}\PYG{p}{(}\PYG{n}{thetaL}\PYG{p}{)}\PYG{o}{*}\PYG{n}{np}\PYG{o}{.}\PYG{n}{sin}\PYG{p}{(}\PYG{n}{theta}\PYG{p}{)}\PYG{p}{)}\PYG{o}{*}\PYG{o}{*}\PYG{p}{(}\PYG{o}{\PYGZhy{}}\PYG{l+m+mi}{1}\PYG{p}{)}
\PYG{n}{plt}\PYG{o}{.}\PYG{n}{plot}\PYG{p}{(}\PYG{n}{theta}\PYG{o}{*}\PYG{l+m+mi}{180}\PYG{o}{/}\PYG{n}{np}\PYG{o}{.}\PYG{n}{pi}\PYG{p}{,} \PYG{n}{F}\PYG{p}{)}
\PYG{n}{plt}\PYG{o}{.}\PYG{n}{ylim}\PYG{p}{(}\PYG{o}{\PYGZhy{}}\PYG{l+m+mf}{0.1}\PYG{p}{,}\PYG{l+m+mf}{0.1}\PYG{p}{)}
\PYG{n}{plt}\PYG{o}{.}\PYG{n}{xticks}\PYG{p}{(}\PYG{n}{np}\PYG{o}{.}\PYG{n}{arange}\PYG{p}{(}\PYG{l+m+mi}{0}\PYG{p}{,}\PYG{l+m+mi}{360}\PYG{p}{,}\PYG{l+m+mi}{30}\PYG{p}{)}\PYG{p}{)}
\PYG{n}{plt}\PYG{o}{.}\PYG{n}{xlabel}\PYG{p}{(}\PYG{l+s+sa}{r}\PYG{l+s+s1}{\PYGZsq{}}\PYG{l+s+s1}{\PYGZdl{}}\PYG{l+s+s1}{\PYGZbs{}}\PYG{l+s+s1}{theta\PYGZdl{} (degrees)}\PYG{l+s+s1}{\PYGZsq{}}\PYG{p}{)}
\PYG{n}{plt}\PYG{o}{.}\PYG{n}{ylabel}\PYG{p}{(}\PYG{l+s+s1}{\PYGZsq{}}\PYG{l+s+s1}{Piston Force}\PYG{l+s+s1}{\PYGZsq{}}\PYG{p}{)}\PYG{p}{;}
\end{sphinxVerbatim}

\noindent\sphinxincludegraphics{{driving-forces_8_0}.png}


\chapter{Module 2 \sphinxhyphen{} constrained dynamic motion}
\label{\detokenize{module_02/overview:module-2-constrained-dynamic-motion}}\label{\detokenize{module_02/overview::doc}}
\sphinxstyleemphasis{pendulum motion}
\begin{enumerate}
\sphinxsetlistlabels{\arabic}{enumi}{enumii}{}{.}%
\item {} 
Types of constraints and degrees of freedom

\item {} 
pin joint

\item {} 
sliding constraint

\item {} 
rigid attachment

\item {} 
Greubler count

\item {} 
Equations of motion with constraints

\item {} 
count degrees of freedom choose variables

\item {} 
define forces and kinematics with chosen variable

\item {} 
solve for constrained equation of motion

\item {} 
Using smart coordinate systems

\item {} 
using mixed unit vectors

\item {} 
radial and spherical

\item {} 
rotating systems

\end{enumerate}


\chapter{Module 3 \sphinxhyphen{} Constrained engineering systems}
\label{\detokenize{module_03/overview:module-3-constrained-engineering-systems}}\label{\detokenize{module_03/overview::doc}}
\sphinxstyleemphasis{4\sphinxhyphen{}bar linkage}
\begin{enumerate}
\sphinxsetlistlabels{\arabic}{enumi}{enumii}{}{.}%
\item {} 
Engineering systems with many moving parts
\begin{enumerate}
\sphinxsetlistlabels{\arabic}{enumii}{enumiii}{}{.}%
\item {} 
constrain the motion (0\sphinxhyphen{}DOF)

\item {} 
to maintain motion need constraint forces

\item {} 
constraints doing work vs constraint force

\end{enumerate}

\item {} 
Mechanical advantage
\begin{enumerate}
\sphinxsetlistlabels{\arabic}{enumii}{enumiii}{}{.}%
\item {} 
gear ratios

\item {} 
levers

\item {} 
linkages

\end{enumerate}

\item {} 
Relative motion
\begin{enumerate}
\sphinxsetlistlabels{\arabic}{enumii}{enumiii}{}{.}%
\item {} 
fixing coordinate systems to moving parts

\item {} 
absolute and relative motion

\item {} 
constraint forces and work done

\end{enumerate}

\end{enumerate}

\begin{sphinxVerbatim}[commandchars=\\\{\}]
\PYG{k+kn}{import} \PYG{n+nn}{numpy} \PYG{k}{as} \PYG{n+nn}{np}
\PYG{k+kn}{import} \PYG{n+nn}{matplotlib}\PYG{n+nn}{.}\PYG{n+nn}{pyplot} \PYG{k}{as} \PYG{n+nn}{plt}
\PYG{k+kn}{from} \PYG{n+nn}{scipy}\PYG{n+nn}{.}\PYG{n+nn}{integrate} \PYG{k+kn}{import} \PYG{n}{odeint}
\PYG{k+kn}{import} \PYG{n+nn}{pretty\PYGZus{}plots} \PYG{c+c1}{\PYGZsh{} script to set up LaTex and increase line\PYGZhy{}width and font size}
\end{sphinxVerbatim}

\begin{sphinxVerbatim}[commandchars=\\\{\}]
\PYG{g+gt}{\PYGZhy{}\PYGZhy{}\PYGZhy{}\PYGZhy{}\PYGZhy{}\PYGZhy{}\PYGZhy{}\PYGZhy{}\PYGZhy{}\PYGZhy{}\PYGZhy{}\PYGZhy{}\PYGZhy{}\PYGZhy{}\PYGZhy{}\PYGZhy{}\PYGZhy{}\PYGZhy{}\PYGZhy{}\PYGZhy{}\PYGZhy{}\PYGZhy{}\PYGZhy{}\PYGZhy{}\PYGZhy{}\PYGZhy{}\PYGZhy{}\PYGZhy{}\PYGZhy{}\PYGZhy{}\PYGZhy{}\PYGZhy{}\PYGZhy{}\PYGZhy{}\PYGZhy{}\PYGZhy{}\PYGZhy{}\PYGZhy{}\PYGZhy{}\PYGZhy{}\PYGZhy{}\PYGZhy{}\PYGZhy{}\PYGZhy{}\PYGZhy{}\PYGZhy{}\PYGZhy{}\PYGZhy{}\PYGZhy{}\PYGZhy{}\PYGZhy{}\PYGZhy{}\PYGZhy{}\PYGZhy{}\PYGZhy{}\PYGZhy{}\PYGZhy{}\PYGZhy{}\PYGZhy{}\PYGZhy{}\PYGZhy{}\PYGZhy{}\PYGZhy{}\PYGZhy{}\PYGZhy{}\PYGZhy{}\PYGZhy{}\PYGZhy{}\PYGZhy{}\PYGZhy{}\PYGZhy{}\PYGZhy{}\PYGZhy{}\PYGZhy{}\PYGZhy{}}
\PYG{n+ne}{ModuleNotFoundError}\PYG{g+gWhitespace}{                       }Traceback (most recent call last)
\PYG{o}{\PYGZlt{}}\PYG{n}{ipython}\PYG{o}{\PYGZhy{}}\PYG{n+nb}{input}\PYG{o}{\PYGZhy{}}\PYG{l+m+mi}{1}\PYG{o}{\PYGZhy{}}\PYG{l+m+mf}{5108677927e6}\PYG{o}{\PYGZgt{}} \PYG{o+ow}{in} \PYG{o}{\PYGZlt{}}\PYG{n}{module}\PYG{o}{\PYGZgt{}}
\PYG{g+gWhitespace}{      }\PYG{l+m+mi}{2} \PYG{k+kn}{import} \PYG{n+nn}{matplotlib}\PYG{n+nn}{.}\PYG{n+nn}{pyplot} \PYG{k}{as} \PYG{n+nn}{plt}
\PYG{g+gWhitespace}{      }\PYG{l+m+mi}{3} \PYG{k+kn}{from} \PYG{n+nn}{scipy}\PYG{n+nn}{.}\PYG{n+nn}{integrate} \PYG{k+kn}{import} \PYG{n}{odeint}
\PYG{n+ne}{\PYGZhy{}\PYGZhy{}\PYGZhy{}\PYGZhy{}\PYGZgt{} }\PYG{l+m+mi}{4} \PYG{k+kn}{import} \PYG{n+nn}{pretty\PYGZus{}plots} \PYG{c+c1}{\PYGZsh{} script to set up LaTex and increase line\PYGZhy{}width and font size}

\PYG{n+ne}{ModuleNotFoundError}: No module named \PYGZsq{}pretty\PYGZus{}plots\PYGZsq{}
\end{sphinxVerbatim}

\begin{sphinxVerbatim}[commandchars=\\\{\}]
\PYG{n}{pretty\PYGZus{}plots}\PYG{o}{.}\PYG{n}{setdefaults}\PYG{p}{(}\PYG{p}{)}
\end{sphinxVerbatim}


\section{Numerical Solution of Yoyo despinning}
\label{\detokenize{module_03/yoyo-despin:numerical-solution-of-yoyo-despinning}}\label{\detokenize{module_03/yoyo-despin::doc}}
\(\ddot{r} = r \dot{\theta}^2\)

\(\ddot{r} = r\left(\frac{2mr_0^2+MR^2/2}{2mr^2+MR^2/2}\right)^2 \dot{\theta}(0)\)

Here we plot the relation between angular velocity and distance from center of cylinder just based upon conservation of angular momentum:

\begin{sphinxVerbatim}[commandchars=\\\{\}]
\PYG{n}{m}\PYG{o}{=}\PYG{l+m+mf}{0.1} \PYG{c+c1}{\PYGZsh{}kg}
\PYG{n}{M}\PYG{o}{=}\PYG{l+m+mi}{1} \PYG{c+c1}{\PYGZsh{}kg}
\PYG{n}{R}\PYG{o}{=}\PYG{l+m+mf}{0.1} \PYG{c+c1}{\PYGZsh{}meter}
\PYG{n}{w0}\PYG{o}{=}\PYG{l+m+mi}{10} \PYG{c+c1}{\PYGZsh{} rad/s}
\PYG{n}{r}\PYG{o}{=}\PYG{n}{np}\PYG{o}{.}\PYG{n}{linspace}\PYG{p}{(}\PYG{l+m+mf}{0.1}\PYG{p}{,}\PYG{l+m+mi}{2}\PYG{p}{)}
\PYG{n}{h0}\PYG{o}{=}\PYG{p}{(}\PYG{l+m+mi}{2}\PYG{o}{*}\PYG{n}{m}\PYG{o}{*}\PYG{n}{r}\PYG{p}{[}\PYG{l+m+mi}{0}\PYG{p}{]}\PYG{o}{*}\PYG{o}{*}\PYG{l+m+mi}{2}\PYG{o}{+}\PYG{n}{M}\PYG{o}{*}\PYG{n}{r}\PYG{p}{[}\PYG{l+m+mi}{0}\PYG{p}{]}\PYG{o}{*}\PYG{o}{*}\PYG{l+m+mi}{2}\PYG{o}{/}\PYG{l+m+mi}{2}\PYG{p}{)}\PYG{o}{*}\PYG{n}{w0}

\PYG{n}{wt} \PYG{o}{=} \PYG{n}{h0}\PYG{o}{/}\PYG{p}{(}\PYG{l+m+mi}{2}\PYG{o}{*}\PYG{n}{m}\PYG{o}{*}\PYG{n}{r}\PYG{o}{*}\PYG{o}{*}\PYG{l+m+mi}{2}\PYG{o}{+}\PYG{n}{M}\PYG{o}{*}\PYG{n}{R}\PYG{o}{*}\PYG{o}{*}\PYG{l+m+mi}{2}\PYG{o}{/}\PYG{l+m+mi}{2}\PYG{p}{)}

\PYG{n}{plt}\PYG{o}{.}\PYG{n}{plot}\PYG{p}{(}\PYG{n}{r}\PYG{p}{,}\PYG{n}{wt}\PYG{p}{)}
\PYG{n}{plt}\PYG{o}{.}\PYG{n}{xlabel}\PYG{p}{(}\PYG{l+s+s1}{\PYGZsq{}}\PYG{l+s+s1}{yoyo distance (m)}\PYG{l+s+s1}{\PYGZsq{}}\PYG{p}{)}
\PYG{n}{plt}\PYG{o}{.}\PYG{n}{ylabel}\PYG{p}{(}\PYG{l+s+s1}{\PYGZsq{}}\PYG{l+s+s1}{angular velocity (rad/s)}\PYG{l+s+s1}{\PYGZsq{}}\PYG{p}{)}
\PYG{n}{plt}\PYG{o}{.}\PYG{n}{title}\PYG{p}{(}\PYG{l+s+s1}{\PYGZsq{}}\PYG{l+s+s1}{conservation of angular momentum}\PYG{l+s+se}{\PYGZbs{}n}\PYG{l+s+s1}{ h=constant}\PYG{l+s+s1}{\PYGZsq{}}\PYG{p}{)}\PYG{p}{;}
\end{sphinxVerbatim}

\noindent\sphinxincludegraphics{{yoyo-despin_4_0}.png}


\section{Define the state and d/dt(state)}
\label{\detokenize{module_03/yoyo-despin:define-the-state-and-d-dt-state}}
In this part, we define the second order differential equation as a state\sphinxhyphen{}space form

the state = \([r,~\dot{r}]\)

and the derivative of the state is

d/dt(state) = \([\dot{r},~\ddot{r}]=[\dot{r},~\frac{F_r}{m}]\)

We call the function, \sphinxcode{\sphinxupquote{yoyo\_ode(y,t)}}, where \sphinxcode{\sphinxupquote{y}} is the state and \sphinxcode{\sphinxupquote{t}} is the current time.

\begin{sphinxVerbatim}[commandchars=\\\{\}]
\PYG{k}{def} \PYG{n+nf}{yoyo\PYGZus{}ode}\PYG{p}{(}\PYG{n}{y}\PYG{p}{,}\PYG{n}{t}\PYG{p}{)}\PYG{p}{:}
    \PYG{l+s+sd}{\PYGZsq{}\PYGZsq{}\PYGZsq{}define d2r/dt2= r*(h0/2m)\PYGZca{}2/(M*R\PYGZca{}2/4m+r\PYGZca{}2)\PYGZca{}2\PYGZsq{}\PYGZsq{}\PYGZsq{}}
    \PYG{n}{dr}\PYG{o}{=}\PYG{n}{np}\PYG{o}{.}\PYG{n}{zeros}\PYG{p}{(}\PYG{n}{np}\PYG{o}{.}\PYG{n}{shape}\PYG{p}{(}\PYG{n}{y}\PYG{p}{)}\PYG{p}{)}
    \PYG{n}{dr}\PYG{p}{[}\PYG{l+m+mi}{0}\PYG{p}{]}\PYG{o}{=}\PYG{n}{y}\PYG{p}{[}\PYG{l+m+mi}{1}\PYG{p}{]}
    \PYG{n}{dr}\PYG{p}{[}\PYG{l+m+mi}{1}\PYG{p}{]}\PYG{o}{=}\PYG{n}{y}\PYG{p}{[}\PYG{l+m+mi}{0}\PYG{p}{]}\PYG{o}{*}\PYG{n}{h0}\PYG{o}{*}\PYG{o}{*}\PYG{l+m+mi}{2}\PYG{o}{/}\PYG{p}{(}\PYG{l+m+mi}{2}\PYG{o}{*}\PYG{n}{m}\PYG{o}{*}\PYG{n}{y}\PYG{p}{[}\PYG{l+m+mi}{0}\PYG{p}{]}\PYG{o}{*}\PYG{o}{*}\PYG{l+m+mi}{2}\PYG{o}{+}\PYG{n}{M}\PYG{o}{*}\PYG{n}{R}\PYG{o}{*}\PYG{o}{*}\PYG{l+m+mi}{2}\PYG{o}{/}\PYG{l+m+mi}{2}\PYG{p}{)}\PYG{o}{*}\PYG{o}{*}\PYG{l+m+mi}{2}
    \PYG{k}{return} \PYG{n}{dr}
\end{sphinxVerbatim}

\begin{sphinxVerbatim}[commandchars=\\\{\}]
\PYG{n}{yoyo\PYGZus{}ode}\PYG{p}{(}\PYG{p}{[}\PYG{l+m+mf}{0.15}\PYG{p}{,}\PYG{l+m+mi}{1}\PYG{p}{]}\PYG{p}{,}\PYG{l+m+mi}{0}\PYG{p}{)}
\end{sphinxVerbatim}

\begin{sphinxVerbatim}[commandchars=\\\{\}]
array([ 1.        ,  8.14404432])
\end{sphinxVerbatim}

The function \sphinxcode{\sphinxupquote{odeint}} integrates our \sphinxcode{\sphinxupquote{yoyo\_ode}} based upon the initial condtions,

\([r(0),~\dot{r}(0)] = [0.1~m,~0~m/s]\)

and the time span of interest,

\(t = [0-0.5~s]\)

in the line, \sphinxcode{\sphinxupquote{t=np.linspace(0,0.5)}}

\begin{sphinxVerbatim}[commandchars=\\\{\}]
\PYG{n}{t}\PYG{o}{=}\PYG{n}{np}\PYG{o}{.}\PYG{n}{linspace}\PYG{p}{(}\PYG{l+m+mi}{0}\PYG{p}{,}\PYG{l+m+mf}{0.5}\PYG{p}{)}
\PYG{n}{r}\PYG{o}{=}\PYG{n}{odeint}\PYG{p}{(}\PYG{n}{yoyo\PYGZus{}ode}\PYG{p}{,}\PYG{p}{[}\PYG{l+m+mf}{0.1}\PYG{p}{,}\PYG{l+m+mi}{0}\PYG{p}{]}\PYG{p}{,}\PYG{n}{t}\PYG{p}{)}

\PYG{n}{plt}\PYG{o}{.}\PYG{n}{plot}\PYG{p}{(}\PYG{n}{t}\PYG{p}{,}\PYG{n}{r}\PYG{p}{[}\PYG{p}{:}\PYG{p}{,}\PYG{l+m+mi}{0}\PYG{p}{]}\PYG{p}{)}
\PYG{n}{plt}\PYG{o}{.}\PYG{n}{xlabel}\PYG{p}{(}\PYG{l+s+s1}{\PYGZsq{}}\PYG{l+s+s1}{time (s)}\PYG{l+s+s1}{\PYGZsq{}}\PYG{p}{)}
\PYG{n}{plt}\PYG{o}{.}\PYG{n}{ylabel}\PYG{p}{(}\PYG{l+s+s1}{\PYGZsq{}}\PYG{l+s+s1}{yoyo pos (m)}\PYG{l+s+s1}{\PYGZsq{}}\PYG{p}{)}
\end{sphinxVerbatim}

\begin{sphinxVerbatim}[commandchars=\\\{\}]
Text(0,0.5,\PYGZsq{}yoyo pos (m)\PYGZsq{})
\end{sphinxVerbatim}

\noindent\sphinxincludegraphics{{yoyo-despin_9_1}.png}


\section{To save to file from python output}
\label{\detokenize{module_03/yoyo-despin:to-save-to-file-from-python-output}}
Have to join the time, \(r\), and \(\dot{r}\) into an array then save to a file:

\begin{sphinxVerbatim}[commandchars=\\\{\}]
\PYG{n}{np}\PYG{o}{.}\PYG{n}{savetxt}\PYG{p}{(}\PYG{l+s+s1}{\PYGZsq{}}\PYG{l+s+s1}{t\PYGZus{}r\PYGZus{}rdot.csv}\PYG{l+s+s1}{\PYGZsq{}}\PYG{p}{,}\PYG{n}{np}\PYG{o}{.}\PYG{n}{array}\PYG{p}{(}\PYG{p}{[}\PYG{n}{t}\PYG{p}{,}\PYG{n}{r}\PYG{p}{[}\PYG{p}{:}\PYG{p}{,}\PYG{l+m+mi}{0}\PYG{p}{]}\PYG{p}{,}\PYG{n}{r}\PYG{p}{[}\PYG{p}{:}\PYG{p}{,}\PYG{l+m+mi}{1}\PYG{p}{]}\PYG{p}{]}\PYG{p}{)}\PYG{o}{.}\PYG{n}{T}\PYG{p}{,}\PYG{n}{delimiter}\PYG{o}{=}\PYG{l+s+s1}{\PYGZsq{}}\PYG{l+s+s1}{,}\PYG{l+s+s1}{\PYGZsq{}}\PYG{p}{)}
\end{sphinxVerbatim}

This line of code organizes a comma\sphinxhyphen{}separated\sphinxhyphen{}value file into the file t\_r\_rdot.csv with no headers.


\begin{savenotes}\sphinxattablestart
\centering
\begin{tabulary}{\linewidth}[t]{|T|T|T|}
\hline
\sphinxstyletheadfamily 
time (s)
&\sphinxstyletheadfamily 
r (m)
&\sphinxstyletheadfamily 
\(\dot{r}\) (m/s)
\\
\hline
0
&
0.1
&
0
\\
\hline
…
&
…
&
…
\\
\hline
\end{tabulary}
\par
\sphinxattableend\end{savenotes}


\chapter{Module 4 \sphinxhyphen{} Momentum and Energy}
\label{\detokenize{module_04/overview:module-4-momentum-and-energy}}\label{\detokenize{module_04/overview::doc}}
\sphinxstyleemphasis{yoyo despin}

This module uses energy and momentum methods directly.
\begin{enumerate}
\sphinxsetlistlabels{\arabic}{enumi}{enumii}{}{.}%
\item {} 
Energy and Momentum are different forms of Newton’s
second law
\begin{enumerate}
\sphinxsetlistlabels{\arabic}{enumii}{enumiii}{}{.}%
\item {} 
Impulse\sphinxhyphen{}momentum is Fdt=dp

\item {} 
Work\sphinxhyphen{}energy is Fdx=dT

\item {} 
momentum is a vector

\item {} 
energy is a scalar

\end{enumerate}

\item {} 
Describes states of system

\item {} 
Impulse happens over short time in impact

\item {} 
energy in conservative system ia conserved

\item {} 
combining methods

\end{enumerate}

\begin{sphinxVerbatim}[commandchars=\\\{\}]
\PYG{k+kn}{import} \PYG{n+nn}{numpy} \PYG{k}{as} \PYG{n+nn}{np}
\PYG{k+kn}{import} \PYG{n+nn}{matplotlib}\PYG{n+nn}{.}\PYG{n+nn}{pyplot} \PYG{k}{as} \PYG{n+nn}{plt}
\PYG{k+kn}{from} \PYG{n+nn}{scipy}\PYG{n+nn}{.}\PYG{n+nn}{integrate} \PYG{k+kn}{import} \PYG{n}{odeint}
\end{sphinxVerbatim}


\section{Numerical Solution of Yoyo despinning}
\label{\detokenize{module_04/yoyo-despin:numerical-solution-of-yoyo-despinning}}\label{\detokenize{module_04/yoyo-despin::doc}}
\(\ddot{r} = r \dot{\theta}^2\)

\(\ddot{r} = r\left(\frac{2mr_0^2+MR^2/2}{2mr^2+MR^2/2}\right)^2 \dot{\theta}(0)\)

Here we plot the relation between angular velocity and distance from center of cylinder just based upon conservation of angular momentum:

\begin{sphinxVerbatim}[commandchars=\\\{\}]
\PYG{n}{m}\PYG{o}{=}\PYG{l+m+mf}{0.1} \PYG{c+c1}{\PYGZsh{}kg}
\PYG{n}{M}\PYG{o}{=}\PYG{l+m+mi}{1} \PYG{c+c1}{\PYGZsh{}kg}
\PYG{n}{R}\PYG{o}{=}\PYG{l+m+mf}{0.1} \PYG{c+c1}{\PYGZsh{}meter}
\PYG{n}{w0}\PYG{o}{=}\PYG{l+m+mi}{10} \PYG{c+c1}{\PYGZsh{} rad/s}
\PYG{n}{r}\PYG{o}{=}\PYG{n}{np}\PYG{o}{.}\PYG{n}{linspace}\PYG{p}{(}\PYG{l+m+mf}{0.1}\PYG{p}{,}\PYG{l+m+mi}{2}\PYG{p}{)}
\PYG{n}{h0}\PYG{o}{=}\PYG{p}{(}\PYG{l+m+mi}{2}\PYG{o}{*}\PYG{n}{m}\PYG{o}{*}\PYG{n}{r}\PYG{p}{[}\PYG{l+m+mi}{0}\PYG{p}{]}\PYG{o}{*}\PYG{o}{*}\PYG{l+m+mi}{2}\PYG{o}{+}\PYG{n}{M}\PYG{o}{*}\PYG{n}{r}\PYG{p}{[}\PYG{l+m+mi}{0}\PYG{p}{]}\PYG{o}{*}\PYG{o}{*}\PYG{l+m+mi}{2}\PYG{o}{/}\PYG{l+m+mi}{2}\PYG{p}{)}\PYG{o}{*}\PYG{n}{w0}

\PYG{n}{wt} \PYG{o}{=} \PYG{n}{h0}\PYG{o}{/}\PYG{p}{(}\PYG{l+m+mi}{2}\PYG{o}{*}\PYG{n}{m}\PYG{o}{*}\PYG{n}{r}\PYG{o}{*}\PYG{o}{*}\PYG{l+m+mi}{2}\PYG{o}{+}\PYG{n}{M}\PYG{o}{*}\PYG{n}{R}\PYG{o}{*}\PYG{o}{*}\PYG{l+m+mi}{2}\PYG{o}{/}\PYG{l+m+mi}{2}\PYG{p}{)}

\PYG{n}{plt}\PYG{o}{.}\PYG{n}{plot}\PYG{p}{(}\PYG{n}{r}\PYG{p}{,}\PYG{n}{wt}\PYG{p}{)}
\PYG{n}{plt}\PYG{o}{.}\PYG{n}{xlabel}\PYG{p}{(}\PYG{l+s+s1}{\PYGZsq{}}\PYG{l+s+s1}{yoyo distance (m)}\PYG{l+s+s1}{\PYGZsq{}}\PYG{p}{)}
\PYG{n}{plt}\PYG{o}{.}\PYG{n}{ylabel}\PYG{p}{(}\PYG{l+s+s1}{\PYGZsq{}}\PYG{l+s+s1}{angular velocity (rad/s)}\PYG{l+s+s1}{\PYGZsq{}}\PYG{p}{)}
\PYG{n}{plt}\PYG{o}{.}\PYG{n}{title}\PYG{p}{(}\PYG{l+s+s1}{\PYGZsq{}}\PYG{l+s+s1}{conservation of angular momentum}\PYG{l+s+se}{\PYGZbs{}n}\PYG{l+s+s1}{ h=constant}\PYG{l+s+s1}{\PYGZsq{}}\PYG{p}{)}\PYG{p}{;}
\end{sphinxVerbatim}

\noindent\sphinxincludegraphics{{yoyo-despin_3_0}.png}


\section{Define the state and d/dt(state)}
\label{\detokenize{module_04/yoyo-despin:define-the-state-and-d-dt-state}}
In this part, we define the second order differential equation as a state\sphinxhyphen{}space form

the state = \([r,~\dot{r}]\)

and the derivative of the state is

d/dt(state) = \([\dot{r},~\ddot{r}]=[\dot{r},~\frac{F_r}{m}]\)

We call the function, \sphinxcode{\sphinxupquote{yoyo\_ode(y,t)}}, where \sphinxcode{\sphinxupquote{y}} is the state and \sphinxcode{\sphinxupquote{t}} is the current time.

\begin{sphinxVerbatim}[commandchars=\\\{\}]
\PYG{k}{def} \PYG{n+nf}{yoyo\PYGZus{}ode}\PYG{p}{(}\PYG{n}{y}\PYG{p}{,}\PYG{n}{t}\PYG{p}{)}\PYG{p}{:}
    \PYG{l+s+sd}{\PYGZsq{}\PYGZsq{}\PYGZsq{}define d2r/dt2= r*(h0/2m)\PYGZca{}2/(M*R\PYGZca{}2/4m+r\PYGZca{}2)\PYGZca{}2\PYGZsq{}\PYGZsq{}\PYGZsq{}}
    \PYG{n}{dr}\PYG{o}{=}\PYG{n}{np}\PYG{o}{.}\PYG{n}{zeros}\PYG{p}{(}\PYG{n}{np}\PYG{o}{.}\PYG{n}{shape}\PYG{p}{(}\PYG{n}{y}\PYG{p}{)}\PYG{p}{)}
    \PYG{n}{dr}\PYG{p}{[}\PYG{l+m+mi}{0}\PYG{p}{]}\PYG{o}{=}\PYG{n}{y}\PYG{p}{[}\PYG{l+m+mi}{1}\PYG{p}{]}
    \PYG{n}{dr}\PYG{p}{[}\PYG{l+m+mi}{1}\PYG{p}{]}\PYG{o}{=}\PYG{n}{y}\PYG{p}{[}\PYG{l+m+mi}{0}\PYG{p}{]}\PYG{o}{*}\PYG{n}{h0}\PYG{o}{*}\PYG{o}{*}\PYG{l+m+mi}{2}\PYG{o}{/}\PYG{p}{(}\PYG{l+m+mi}{2}\PYG{o}{*}\PYG{n}{m}\PYG{o}{*}\PYG{n}{y}\PYG{p}{[}\PYG{l+m+mi}{0}\PYG{p}{]}\PYG{o}{*}\PYG{o}{*}\PYG{l+m+mi}{2}\PYG{o}{+}\PYG{n}{M}\PYG{o}{*}\PYG{n}{R}\PYG{o}{*}\PYG{o}{*}\PYG{l+m+mi}{2}\PYG{o}{/}\PYG{l+m+mi}{2}\PYG{p}{)}\PYG{o}{*}\PYG{o}{*}\PYG{l+m+mi}{2}
    \PYG{k}{return} \PYG{n}{dr}
\end{sphinxVerbatim}

\begin{sphinxVerbatim}[commandchars=\\\{\}]
\PYG{n}{yoyo\PYGZus{}ode}\PYG{p}{(}\PYG{p}{[}\PYG{l+m+mf}{0.15}\PYG{p}{,}\PYG{l+m+mi}{1}\PYG{p}{]}\PYG{p}{,}\PYG{l+m+mi}{0}\PYG{p}{)}
\end{sphinxVerbatim}

\begin{sphinxVerbatim}[commandchars=\\\{\}]
array([1.        , 8.14404432])
\end{sphinxVerbatim}

The function \sphinxcode{\sphinxupquote{odeint}} integrates our \sphinxcode{\sphinxupquote{yoyo\_ode}} based upon the initial condtions,

\([r(0),~\dot{r}(0)] = [0.1~m,~0~m/s]\)

and the time span of interest,

\(t = [0-0.5~s]\)

in the line, \sphinxcode{\sphinxupquote{t=np.linspace(0,0.5)}}

\begin{sphinxVerbatim}[commandchars=\\\{\}]
\PYG{n}{t}\PYG{o}{=}\PYG{n}{np}\PYG{o}{.}\PYG{n}{linspace}\PYG{p}{(}\PYG{l+m+mi}{0}\PYG{p}{,}\PYG{l+m+mf}{0.5}\PYG{p}{)}
\PYG{n}{r}\PYG{o}{=}\PYG{n}{odeint}\PYG{p}{(}\PYG{n}{yoyo\PYGZus{}ode}\PYG{p}{,}\PYG{p}{[}\PYG{l+m+mf}{0.1}\PYG{p}{,}\PYG{l+m+mi}{0}\PYG{p}{]}\PYG{p}{,}\PYG{n}{t}\PYG{p}{)}

\PYG{n}{plt}\PYG{o}{.}\PYG{n}{plot}\PYG{p}{(}\PYG{n}{t}\PYG{p}{,}\PYG{n}{r}\PYG{p}{[}\PYG{p}{:}\PYG{p}{,}\PYG{l+m+mi}{0}\PYG{p}{]}\PYG{p}{)}
\PYG{n}{plt}\PYG{o}{.}\PYG{n}{xlabel}\PYG{p}{(}\PYG{l+s+s1}{\PYGZsq{}}\PYG{l+s+s1}{time (s)}\PYG{l+s+s1}{\PYGZsq{}}\PYG{p}{)}
\PYG{n}{plt}\PYG{o}{.}\PYG{n}{ylabel}\PYG{p}{(}\PYG{l+s+s1}{\PYGZsq{}}\PYG{l+s+s1}{yoyo pos (m)}\PYG{l+s+s1}{\PYGZsq{}}\PYG{p}{)}
\end{sphinxVerbatim}

\begin{sphinxVerbatim}[commandchars=\\\{\}]
Text(0, 0.5, \PYGZsq{}yoyo pos (m)\PYGZsq{})
\end{sphinxVerbatim}

\noindent\sphinxincludegraphics{{yoyo-despin_8_1}.png}


\section{To save to file from python output}
\label{\detokenize{module_04/yoyo-despin:to-save-to-file-from-python-output}}
Have to join the time, \(r\), and \(\dot{r}\) into an array then save to a file:

\begin{sphinxVerbatim}[commandchars=\\\{\}]
\PYG{n}{np}\PYG{o}{.}\PYG{n}{savetxt}\PYG{p}{(}\PYG{l+s+s1}{\PYGZsq{}}\PYG{l+s+s1}{t\PYGZus{}r\PYGZus{}rdot.csv}\PYG{l+s+s1}{\PYGZsq{}}\PYG{p}{,}\PYG{n}{np}\PYG{o}{.}\PYG{n}{array}\PYG{p}{(}\PYG{p}{[}\PYG{n}{t}\PYG{p}{,}\PYG{n}{r}\PYG{p}{[}\PYG{p}{:}\PYG{p}{,}\PYG{l+m+mi}{0}\PYG{p}{]}\PYG{p}{,}\PYG{n}{r}\PYG{p}{[}\PYG{p}{:}\PYG{p}{,}\PYG{l+m+mi}{1}\PYG{p}{]}\PYG{p}{]}\PYG{p}{)}\PYG{o}{.}\PYG{n}{T}\PYG{p}{,}\PYG{n}{delimiter}\PYG{o}{=}\PYG{l+s+s1}{\PYGZsq{}}\PYG{l+s+s1}{,}\PYG{l+s+s1}{\PYGZsq{}}\PYG{p}{)}
\end{sphinxVerbatim}

This line of code organizes a comma\sphinxhyphen{}separated\sphinxhyphen{}value file into the file t\_r\_rdot.csv with no headers.


\begin{savenotes}\sphinxattablestart
\centering
\begin{tabulary}{\linewidth}[t]{|T|T|T|}
\hline
\sphinxstyletheadfamily 
time (s)
&\sphinxstyletheadfamily 
r (m)
&\sphinxstyletheadfamily 
\(\dot{r}\) (m/s)
\\
\hline
0
&
0.1
&
0
\\
\hline
…
&
…
&
…
\\
\hline
\end{tabulary}
\par
\sphinxattableend\end{savenotes}

\begin{sphinxVerbatim}[commandchars=\\\{\}]
\PYG{k+kn}{import} \PYG{n+nn}{numpy} \PYG{k}{as} \PYG{n+nn}{np}
\PYG{k+kn}{import} \PYG{n+nn}{matplotlib}\PYG{n+nn}{.}\PYG{n+nn}{pyplot} \PYG{k}{as} \PYG{n+nn}{plt}
\PYG{k+kn}{from} \PYG{n+nn}{scipy}\PYG{n+nn}{.}\PYG{n+nn}{integrate} \PYG{k+kn}{import} \PYG{n}{odeint}
\PYG{n}{plt}\PYG{o}{.}\PYG{n}{style}\PYG{o}{.}\PYG{n}{use}\PYG{p}{(}\PYG{l+s+s1}{\PYGZsq{}}\PYG{l+s+s1}{fivethirtyeight}\PYG{l+s+s1}{\PYGZsq{}}\PYG{p}{)}
\end{sphinxVerbatim}


\section{Yoyo despin revisited (cord constraint)}
\label{\detokenize{module_04/yoyo-despin_02:yoyo-despin-revisited-cord-constraint}}\label{\detokenize{module_04/yoyo-despin_02::doc}}
\begin{sphinxVerbatim}[commandchars=\\\{\}]
\PYG{k+kn}{from} \PYG{n+nn}{IPython}\PYG{n+nn}{.}\PYG{n+nn}{core}\PYG{n+nn}{.}\PYG{n+nn}{display} \PYG{k+kn}{import} \PYG{n}{SVG}

\PYG{n}{SVG}\PYG{p}{(}\PYG{n}{filename}\PYG{o}{=}\PYG{l+s+s1}{\PYGZsq{}}\PYG{l+s+s1}{./yoyo\PYGZhy{}rocket.svg}\PYG{l+s+s1}{\PYGZsq{}}\PYG{p}{)}
\end{sphinxVerbatim}

\begin{sphinxVerbatim}[commandchars=\\\{\}]
\PYGZlt{}IPython.core.display.SVG object\PYGZgt{}
\end{sphinxVerbatim}

A rocket yoyo\sphinxhyphen{}despinning mechanism uses cords wrapped around the
payload. These cords unravel and slow the spinning of the rocket. In
this tutorial, you will consider the engineering system, conservation of
angular momentum, and conservation of energy.


\section{Engineering system \sphinxhyphen{} kinematics}
\label{\detokenize{module_04/yoyo-despin_02:engineering-system-kinematics}}
As the yoyo mass unravels, it moves further from the payload. The total
distance from the payload center of mass (COM) is described by

\(\mathbf{r}_{P/G} = R\hat{e}_R + l\hat{e}_{\theta}\)

where \(R\) is the payload radius, \(l\) is the length of the cord, and
\(\hat{e}_R\) and \(\hat{e}_{\theta}\) are unit vectors in a cylindrical
coordinate system. The length of the cord depends upon the angle of the
payload, \(\theta\). Consider a spool of thread rolling across the floor,
the thread left on the floor is equal to distance traveled or,
\(R\theta\). Now, the position of yoyo P is

\(\mathbf{r}_{P/G} = R\hat{e}_R + R\theta\hat{e}_{\theta}\)

where \(\theta\) is the change in angle of the payload after the yoyos are
released. The velocity of mass P is \(\dot{\mathbf{r}}_{P/G}\), using the transport
equation

\(\mathbf{v}_{P/G} = \frac{d}{dt}(R\hat{e}_R + R\theta\hat{e}_{\theta}) +
{}^I\mathbf{\omega}^C \times(R\hat{e}_R + R\theta\hat{e}_{\theta})\)

where the total angular velocity is the combination of the payload’s
angular velocity \(\mathbf{\omega}_B\) and the angular velocity of the
yoyo relative to the payload, \({}^B \mathbf{\omega}^C=\dot{\theta}\hat{k}\). The addition of
payload and yoyo angular velocity is the total

\({}^I\mathbf{\omega}^C = \omega_B \hat{k} +
\dot{\theta}\hat{k}\)

\begin{sphinxadmonition}{note}{Note:}
Consider a 1\sphinxhyphen{}kg payload with radius \(R=0.1~m\) and two yoyos of mass,
\(m_P=m_Q=0.1~kg\). The system is released at \(t = 0~s\) when the payload
is spinning at \(\omega_B^0=10~rad/s\).
\end{sphinxadmonition}

\begin{sphinxVerbatim}[commandchars=\\\{\}]
\PYG{c+c1}{\PYGZsh{} Set up the Python variables}
\PYG{n}{M} \PYG{o}{=} \PYG{l+m+mi}{1}
\PYG{n}{R} \PYG{o}{=} \PYG{l+m+mf}{0.1}
\PYG{n}{I} \PYG{o}{=} \PYG{n}{M}\PYG{o}{*}\PYG{n}{R}\PYG{o}{*}\PYG{o}{*}\PYG{l+m+mi}{2}\PYG{o}{/}\PYG{l+m+mi}{2}
\PYG{n}{m} \PYG{o}{=} \PYG{l+m+mf}{0.1}
\PYG{n}{w0} \PYG{o}{=} \PYG{l+m+mi}{10}
\end{sphinxVerbatim}

The yoyos P and Q will remain anti\sphinxhyphen{}symmetric if released at the same time, so

\(\mathbf{r}_{P/G} = -\mathbf{r}_{Q/G},~|\mathbf{r}_{P/G}| =
|\mathbf{r}_{Q/G}|\)

and

\(\mathbf{v}_{P/G} = -\mathbf{v}_{Q/G},~v_P = v_Q.\)

The equations for position and velocity are the essential kinematic
equations to define angular momentum and kinetic energy, the two kinetic
equations.


\section{Engineering system \sphinxhyphen{} kinetics (conservation of angular momentum)}
\label{\detokenize{module_04/yoyo-despin_02:engineering-system-kinetics-conservation-of-angular-momentum}}
The angular momentum is constant because \(\sum \mathbf{M}_G = 0 =
\frac{d}{dt}\mathbf{h}_G\). The total angular momentum is as such

\(\mathbf{h}_G = I_G \omega_B \hat{k} + m_P \mathbf{r}_{P/G} \times
\mathbf{r}_{P/G}+
 m_Q \mathbf{r}_{Q/G} \times \mathbf{r}_{Q/G}\)

where \(m_P = m_Q = m\)

\(\mathbf{h}_G = I_G \omega_B \hat{k} + 2mR^2(\omega_B +
\theta^2(\omega_B+\dot{\theta})) = (I_G + 2mR^2)\omega_B^0\)

where \(\omega_B^0\) is the initial spinning rate of the payload,
\(\omega_B\) is the angular speed of the payload after the yoyos are
released, \(\theta\) is the angle of the yoyo cord, and \(\dot{\theta}\) is
the relative angular velocity of the yoyo cord. At time \(t=0\), the yoyos
are release and \(\theta = \dot{\theta}=0\). There are three unknown
variables in this angular momentum equation, \(\omega_B\), \(\theta\), and
\(\dot{\theta}\). You need another equation, use work\sphinxhyphen{}energy.


\section{Engineering system \sphinxhyphen{} kinetics (work\sphinxhyphen{}energy formulation)}
\label{\detokenize{module_04/yoyo-despin_02:engineering-system-kinetics-work-energy-formulation}}
There is no external work done to the system, the tension in the yoyo
cords are internal constraint forces, so the total work done is 0, e.g.
for a \(T\Delta l\) there is an equal and opposite \(-T\Delta l\) on
the other side.

\(T_1 + W_{1\rightarrow2} = T_2\)

\(\frac{1}{2}I_G (\omega_B^0)^2 + 2mR^2(\omega_B^0)^2 = 
\frac{1}{2}I_G (\omega_B)^2 +
m(R^2\theta^2(\omega_B+\dot{\theta})^2+R^2\omega_B^2)\)

combining terms and simplifying

\(\left(\frac{I_G}{2mR^2}+1\right)( (\omega_B^0)^2-\omega_B)  = 
\theta^2(\omega_B+\dot{\theta})^2\)


\section{Combining equations and solving}
\label{\detokenize{module_04/yoyo-despin_02:combining-equations-and-solving}}
Substitute \(c = \left(\frac{I_G}{2mR^2}+1\right)\) so you are left with
\begin{enumerate}
\sphinxsetlistlabels{\arabic}{enumi}{enumii}{}{.}%
\item {} 
conservation of angular momentum

\end{enumerate}

\(c(\omega_B^0 - \omega_B) = \theta^2(\omega_B+\dot{\theta})\)
\begin{enumerate}
\sphinxsetlistlabels{\arabic}{enumi}{enumii}{}{.}%
\item {} 
work\sphinxhyphen{}energy

\end{enumerate}

\(c(\omega_B^0-\omega_B^2)(\omega_B^0+\omega_B^2) = \theta^2(\omega_B+\theta)^2\)

dividing work\sphinxhyphen{}energy by conservation of angular momentum,

\(\frac{c(\omega_B^0-\omega_B^2)(\omega_B^0+\omega_B^2)}{c(\omega_B^0 - \omega_B)} = 
\frac{\theta^2(\omega_B+\theta)^2}{\theta^2(\omega_B+\theta)}\)

with the solution for \(\dot{\theta}\)

\(\omega_B^0 +\omega_B = \omega_B +\dot{\theta} \rightarrow \omega_B^0 =
\dot{\theta}\).

\begin{sphinxVerbatim}[commandchars=\\\{\}]
\PYG{n}{t} \PYG{o}{=} \PYG{n}{np}\PYG{o}{.}\PYG{n}{linspace}\PYG{p}{(}\PYG{l+m+mi}{0}\PYG{p}{,}\PYG{l+m+mi}{1}\PYG{p}{)}
\PYG{n}{theta} \PYG{o}{=} \PYG{n}{w0}\PYG{o}{*}\PYG{n}{t} 
\end{sphinxVerbatim}

This result, \sphinxstyleemphasis{combining conservation of angular momentum and
work\sphinxhyphen{}energy}, tells you that the angular velocity of the yoyos will be
be equal to the initial angular velocity of the payload. The angle
\(\theta\) will
continue increase as \(\omega_B^0 t\) until released. Plug this result into
the orginal conservation of angular momentum equation to solve for
\(\omega_B\)

\(c(\omega_B^0 - \omega_B) = (\omega_B^0 t)^2(\omega_B+\omega_B^0)\)

\(\omega_B(t) = \frac{c-(\omega_B^0 t)^2}{c+(\omega_B^0 t)^2}\omega_B^0.\)

\begin{sphinxVerbatim}[commandchars=\\\{\}]
\PYG{n}{c} \PYG{o}{=} \PYG{n}{np}\PYG{o}{.}\PYG{n}{sqrt}\PYG{p}{(}\PYG{n}{I}\PYG{o}{/}\PYG{l+m+mi}{2}\PYG{o}{/}\PYG{n}{m}\PYG{o}{/}\PYG{n}{R}\PYG{o}{*}\PYG{o}{*}\PYG{l+m+mi}{2} \PYG{o}{+} \PYG{l+m+mi}{1}\PYG{p}{)}
\PYG{n}{wB} \PYG{o}{=} \PYG{k}{lambda} \PYG{n}{t}\PYG{p}{:} \PYG{p}{(}\PYG{n}{c}\PYG{o}{\PYGZhy{}}\PYG{n}{w0}\PYG{o}{*}\PYG{o}{*}\PYG{l+m+mi}{2}\PYG{o}{*}\PYG{n}{t}\PYG{o}{*}\PYG{o}{*}\PYG{l+m+mi}{2}\PYG{p}{)}\PYG{o}{/}\PYG{p}{(}\PYG{n}{c}\PYG{o}{+}\PYG{n}{w0}\PYG{o}{*}\PYG{o}{*}\PYG{l+m+mi}{2}\PYG{o}{*}\PYG{n}{t}\PYG{o}{*}\PYG{o}{*}\PYG{l+m+mi}{2}\PYG{p}{)}\PYG{o}{*}\PYG{n}{w0}
\PYG{n}{wC} \PYG{o}{=} \PYG{n}{wB}\PYG{p}{(}\PYG{n}{t}\PYG{p}{)} \PYG{o}{+} \PYG{n}{w0}
\PYG{n}{x} \PYG{o}{=} \PYG{n}{R}\PYG{o}{*}\PYG{n}{np}\PYG{o}{.}\PYG{n}{cos}\PYG{p}{(}\PYG{n}{wC}\PYG{o}{*}\PYG{n}{t}\PYG{p}{)} \PYG{o}{\PYGZhy{}} \PYG{n}{R}\PYG{o}{*}\PYG{n}{w0}\PYG{o}{*}\PYG{n}{t}\PYG{o}{*}\PYG{n}{np}\PYG{o}{.}\PYG{n}{sin}\PYG{p}{(}\PYG{n}{wC}\PYG{o}{*}\PYG{n}{t}\PYG{p}{)}
\PYG{n}{y} \PYG{o}{=} \PYG{n}{R}\PYG{o}{*}\PYG{n}{np}\PYG{o}{.}\PYG{n}{sin}\PYG{p}{(}\PYG{n}{wC}\PYG{o}{*}\PYG{n}{t}\PYG{p}{)} \PYG{o}{+} \PYG{n}{R}\PYG{o}{*}\PYG{n}{w0}\PYG{o}{*}\PYG{n}{t}\PYG{o}{*}\PYG{n}{np}\PYG{o}{.}\PYG{n}{cos}\PYG{p}{(}\PYG{n}{wC}\PYG{o}{*}\PYG{n}{t}\PYG{p}{)}
\PYG{n}{plt}\PYG{o}{.}\PYG{n}{plot}\PYG{p}{(}\PYG{n}{x}\PYG{p}{,}\PYG{n}{y}\PYG{p}{,} \PYG{l+s+s1}{\PYGZsq{}}\PYG{l+s+s1}{o}\PYG{l+s+s1}{\PYGZsq{}}\PYG{p}{,} \PYG{n}{label} \PYG{o}{=} \PYG{l+s+s1}{\PYGZsq{}}\PYG{l+s+s1}{yoyo P}\PYG{l+s+s1}{\PYGZsq{}}\PYG{p}{)}
\PYG{n}{plt}\PYG{o}{.}\PYG{n}{plot}\PYG{p}{(}\PYG{o}{\PYGZhy{}}\PYG{n}{x}\PYG{p}{,}\PYG{o}{\PYGZhy{}}\PYG{n}{y}\PYG{p}{,} \PYG{l+s+s1}{\PYGZsq{}}\PYG{l+s+s1}{o}\PYG{l+s+s1}{\PYGZsq{}}\PYG{p}{,} \PYG{n}{label} \PYG{o}{=} \PYG{l+s+s1}{\PYGZsq{}}\PYG{l+s+s1}{yoyo Q}\PYG{l+s+s1}{\PYGZsq{}}\PYG{p}{)}
\PYG{n}{plt}\PYG{o}{.}\PYG{n}{axis}\PYG{p}{(}\PYG{l+s+s1}{\PYGZsq{}}\PYG{l+s+s1}{equal}\PYG{l+s+s1}{\PYGZsq{}}\PYG{p}{)}
\PYG{n}{plt}\PYG{o}{.}\PYG{n}{title}\PYG{p}{(}\PYG{l+s+s1}{\PYGZsq{}}\PYG{l+s+s1}{paths of yoyos P and Q}\PYG{l+s+s1}{\PYGZsq{}}\PYG{p}{)}
\PYG{n}{plt}\PYG{o}{.}\PYG{n}{xlabel}\PYG{p}{(}\PYG{l+s+s1}{\PYGZsq{}}\PYG{l+s+s1}{x\PYGZhy{}position (m)}\PYG{l+s+s1}{\PYGZsq{}}\PYG{p}{)}
\PYG{n}{plt}\PYG{o}{.}\PYG{n}{ylabel}\PYG{p}{(}\PYG{l+s+s1}{\PYGZsq{}}\PYG{l+s+s1}{y\PYGZhy{}position (m)}\PYG{l+s+s1}{\PYGZsq{}}\PYG{p}{)}
\PYG{n}{plt}\PYG{o}{.}\PYG{n}{legend}\PYG{p}{(}\PYG{p}{)}\PYG{p}{;}
\end{sphinxVerbatim}

\noindent\sphinxincludegraphics{{yoyo-despin_02_9_0}.png}

The added benefit of using cords to release the yoyos is that the
payload angular velocity can be reduced to 0 rad/s at time, \(t_f\).

\(\omega_B(t_f) = 0 = c - (\omega_B^0t)^2 \rightarrow t_f =
\frac{\sqrt{c}}{\omega_B^0} =
\frac{1}{\omega_B^0}\sqrt{\frac{I}{2mR^2}+1}.\)

The final cord length, is unraveling distance, \(l_F = R\theta  =
R\omega_B^0 t_f\)

\(l_F = \sqrt{\frac{I}{2m}+R^2}\)

\begin{sphinxVerbatim}[commandchars=\\\{\}]
\PYG{n}{tf} \PYG{o}{=} \PYG{n}{np}\PYG{o}{.}\PYG{n}{sqrt}\PYG{p}{(}\PYG{n}{c}\PYG{p}{)}\PYG{o}{/}\PYG{n}{w0}
\PYG{n}{lf} \PYG{o}{=} \PYG{n}{R}\PYG{o}{*}\PYG{n}{np}\PYG{o}{.}\PYG{n}{sqrt}\PYG{p}{(}\PYG{n}{c}\PYG{p}{)}
\PYG{n}{plt}\PYG{o}{.}\PYG{n}{plot}\PYG{p}{(}\PYG{n}{t}\PYG{p}{,}\PYG{n}{wB}\PYG{p}{(}\PYG{n}{t}\PYG{p}{)}\PYG{p}{,}\PYG{n}{label} \PYG{o}{=} \PYG{l+s+s1}{\PYGZsq{}}\PYG{l+s+s1}{w\PYGZus{}B(t) solution}\PYG{l+s+s1}{\PYGZsq{}}\PYG{p}{)}
\PYG{n}{plt}\PYG{o}{.}\PYG{n}{plot}\PYG{p}{(}\PYG{n}{tf}\PYG{p}{,}\PYG{n}{wB}\PYG{p}{(}\PYG{n}{tf}\PYG{p}{)}\PYG{p}{,}\PYG{l+s+s1}{\PYGZsq{}}\PYG{l+s+s1}{o}\PYG{l+s+s1}{\PYGZsq{}}\PYG{p}{,} \PYG{n}{label} \PYG{o}{=} \PYG{l+s+s1}{\PYGZsq{}}\PYG{l+s+s1}{release point, t\PYGZus{}f}\PYG{l+s+s1}{\PYGZsq{}}\PYG{p}{)}
\PYG{n}{plt}\PYG{o}{.}\PYG{n}{legend}\PYG{p}{(}\PYG{p}{)}\PYG{p}{;}
\PYG{n}{plt}\PYG{o}{.}\PYG{n}{xlabel}\PYG{p}{(}\PYG{l+s+s1}{\PYGZsq{}}\PYG{l+s+s1}{time (s)}\PYG{l+s+s1}{\PYGZsq{}}\PYG{p}{)}
\PYG{n}{plt}\PYG{o}{.}\PYG{n}{ylabel}\PYG{p}{(}\PYG{l+s+s1}{\PYGZsq{}}\PYG{l+s+s1}{payload angular velocity (rad/s)}\PYG{l+s+s1}{\PYGZsq{}}\PYG{p}{)}\PYG{p}{;}
\end{sphinxVerbatim}

\noindent\sphinxincludegraphics{{yoyo-despin_02_11_0}.png}


\chapter{Module 5 \sphinxhyphen{} dynamics in three dimensions}
\label{\detokenize{module_05/overview:module-5-dynamics-in-three-dimensions}}\label{\detokenize{module_05/overview::doc}}
\sphinxstyleemphasis{spinning top design}
\begin{enumerate}
\sphinxsetlistlabels{\arabic}{enumi}{enumii}{}{.}%
\item {} 
3D Newton\sphinxhyphen{}Euler equations
\begin{enumerate}
\sphinxsetlistlabels{\arabic}{enumii}{enumiii}{}{.}%
\item {} 
6 DOFs \sphinxhyphen{} 6 equations of motion

\item {} 
angular momentum in 3D

\item {} 
\end{enumerate}

\end{enumerate}







\renewcommand{\indexname}{Index}
\printindex
\end{document}